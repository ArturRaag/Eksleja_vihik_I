\begin{center}
\fbox{\fbox{\parbox{6.5in}{\centering
\begin{flushleft}

\vspace{2mm}
\hspace{5mm}
\textbf{\underline{Pythagorase teoreem}}


\vspace{5mm}
\hspace{5mm}
\begin{tikzpicture}[scale=0.5]

\coordinate (A) at (0,0);
\coordinate (B) at (8,0);
\coordinate (C) at (8,4);

\draw[line width= 0.4mm] (A)--(B)--(C)--cycle;

\path (A)--(B) coordinate[pos=0.5](a);
\path (B)--(C) coordinate[pos=0.5](b);
\path (C)--(A) coordinate[pos=0.5](c);

\node[below] at (a){$a$};
\node[right] at (b){$b$};
\node[above] at (c){$c$};

\node[left] at (A){$A$};
\node[right] at (B){$B$};
\node[above] at (C){$C$};

\pic [draw, angle radius = 5mm ] {angle=C--B--A} node at (7.6,0.4){$\cdot$};
\pic [draw, angle radius= 10mm] {angle=B--A--C} node at (1.2,0.3){$\alpha$};
\pic [draw, angle radius = 7mm] {angle = A--C--B} node at (7.6,3.2){$\beta$};

\node[text width= 10cm] at (20,2){\textbf{Hüpotenuus} on täisnurga vastaskülg (joonisel tähistatud $c$ - ga).\\
Täisnurga lähiskülgi nimetatakse \textbf{kaatetiteks} (joonisel tähistatud külgedega $a$ ja $b$).\\

\vspace{2mm}
\textbf{Pythagorase teoreem}\\
Kaatetite ruutude summa on võrdne hüpotenuusi ruuduga.\\
Ehk:};
\end{tikzpicture}


\begin{equation}
\label{40_eq1}
\boxed{c^{2}=a^{2}+b^{2}}
\end{equation}

\hspace{5mm} 
Järelikult, selleks et leida hüpotenuusi pikkust, tuleb avaldise mõlemad pooled võtta ruutjuure alla.

\[ \sqrt{c^{2}}=\sqrt{a^{2}+b^{2}}  \]

\hspace{5mm}
Siit saame, et hüpotenuusi pikkus avaldub järgmiselt:

\begin{equation}
\label{40_eq2}
\boxed{c=\sqrt{a^{2}+b^{2}}}
\end{equation}

\hspace{5mm}
Avaldises \ref{40_eq1} tähtede ümberpaigutamisel, saame kätte ka üldised valemid kaatetite jaoks, mis näevad\\ \hspace{5mm} välja nõnda:

\begin{equation}
\label{40_eq3}
\boxed{
\begin{tabular}{l}
$a=\sqrt{c^{2}-b^{2}}$\\
$b=\sqrt{c^{2}-a^{2}}$
\end{tabular}
}
\end{equation}

\vspace{2mm}
\hspace{5mm}
\textbf{Näide:}

\vspace{2mm}
\hspace{5mm}
Olgu täisnurkse kolmnurga hüpotenuus $25$ cm ning ühe kateeti pikkus $10$ cm. Leia täisnurkse\\ \hspace{5mm} kolmnurga teine kaatet.

\vspace{2mm}
\hspace{5mm}
Ülesandest lugesime välja, et $c=25$ cm ning üks kaatetitest on $a=10$ cm. Kasutame Pythagorase\\ \hspace{5mm} teoreemi \ref{40_eq1}, et leida puudu olev kaatet.

\vspace{2mm}
\hspace{5mm}
\[ 25^{2} = 10^{2}+b^{2} \]

\vspace{2mm}
\hspace{5mm}
Jätame $b^{2}$ paremalepoole võrdust ning ülejäänud arvud viime vasakulepoole.

\[ 25^{2}-10^{2}=b^{2}\]

\vspace{2mm}
\hspace{5mm}
Kuna me soovime leida lihtsalt $b$, mitte $b^{2}$, siis tuleb meil mõlemad võrduse pooled viia ruutjuure alla.\\ \hspace{5mm} Saame:

\[ \sqrt{25^{2}-10^{2}}=\sqrt{b^{2}} \hspace{3mm} \longrightarrow \hspace{3mm} \sqrt{625-100}=b\]

\[b=\sqrt{525} \approx 22.9 (cm) \]
\end{flushleft}
}}}
\end{center}

\pagebreak
\vspace{0.5cm}

\textbf{Märkmed}\\
\vspace{2mm}
\begin{mdframed}[style=graphpaper]
\vspace{22cm}
\end{mdframed}