% ... some material
\pagecolor{black}
\color{white}

\thispagestyle{empty}

%\afterpage{
%\newgeometry{left=7cm,right=7cm, top=4cm}


%Inimese vaim ning tahe elus püsida on üks romantilisemaid nähtusi siin taeva all. Jõuda järelduseni, et elu on enamjaolt traagiline kannatamine ei nõua pikka arutelu. Me oleme oma loomult abitud ning meil jääb ka mulje, et universum püüab meist igal moel lahti saada. Oma kehaehituse poolest jääme me peaagu kõikidele elukatele alla. Meil napib vajalik jõud või mass, et igasuguste kiskjatega sammu pidada. Meil on õhuke nahk, millest lihtne läbi murda. Või kui massi ja jõudu oleks küllaga, leiaks me taas vaevalt märgatavaid ohte maapinna lähedal roomamas, kes meid oma mürgise torkega või puudutusega maatasa teeks.
%Nõnda jääb inimene säärases hierarhias peaaegu kõikidele elukatele alla, nii pisematele, kui ka suurematele. Enda veendmiseks piisab ka sellest, et kui inimene elus püsimise nimel vaeva ei näe, siis lakkab ta olemast. Surelikud oleme kõik, armu meile ei anta. Kuid sellest hoolimata, oleme me ennast uhkelt üleval pidanud. Me seadsime ennast eelnevalt mainitud hierarhiasse domineerivalt sisse. Elukad keda pimedas kartsime, meile enam hirmu ei pakku. Ükski tõvi ega viirus meil ravimatuks jäi. Meie keskmine eluiga on kasvanud märkimisväärsele tasemele. Kuidas me selliste vägitegudeni jõudsime? Kõik on tänu keelele ja abstraktsele mõttele. Esimene aitas meil olla rohkem struktureeritud, organiseeritud ja täpsem. Samuti õnnistas see meid võimega jagada üksteisega oma elukogemusi ja tarkusi, õpetada üksteisele kasulike oskusi, mis omakorda kiirendas inimkonna tõusu. Viimane, mis vaikimisi oli keele produkt, on aidanud meil ennetada erinevaid tragöödiaid. See on ennetanud kiskjaid meid jahtimast, tõvesid meid kurnamast, nälga meid näljutamast. Kogu looduse raevu suudame me vastu pidada ning relvituks teha, mõeldes vaid ette ning tehes koostööd. Ning mõelnud me oleme! Me teame, et inimkonna eluiga sellel Maakeral on piiratud. Tuleb päev mil säravaimad hinged loovad teekonna universumi avarustesse ning garanteerivad ka nõnda, et inimkonna vaim säraks taevakehade kustumiseni. Ning kord kui me jõuame kõrgusteni, kus isegi jumalikud taevad meie silmadele tuvastamatuks jäävad, olgu siis kõigile teada, et maailmas ei ole imesid! On vaid imepärased inimesed.

%\begin{center}
%Artur Raag\\
%2021
%\end{center}


%\clearpage
%\restoregeometry
%} % end of \afterpage{...} material
% ... still more material


\nopagecolor