\begin{center}
\fbox{\fbox{\parbox{6.5in}{\centering
\begin{flushleft}

\hspace{5mm}
\textbf{\underline{Tekstavaldiste kirjutamine}}

\vspace{5mm}
\hspace{5mm}
Igat tundmatut suurust (arvu) võib tähistada suvalise sümboliga. Tavaliselt tähistatakse tundmatuid\\ \hspace{5mm} tähtedega $\mathbf{x, y, z}$ jne.\\
\hspace{5mm}
Antud õppevahendis piirdume vaid juhtudega, kus vajame vaid ühte tundmatut. Kui on kahtlus, et\\ \hspace{5mm} tundmatuid peaks olema rohkem kui üks, siis veendu kõigepealt, et neid ei saa $x$ kaudu avaldada.

\vspace{2mm}
\hspace{5mm}
Enne näidete uurimist tasuks vaadata üle põhilised tekstavaldiste sõnastused.

\vspace{2mm}
\hspace{5mm} 
\begin{tabular}{| c | c |}
	\hline
	Võrra suurem & $+$ \\
	\hline
	Võrra väiksem & $-$ \\
	\hline
	Korda suurem & $\cdot$ \\
	\hline
	Korda väiksem & $:$ \\
	\hline
\end{tabular}
\hspace{0.5cm} 
\begin{tabular}{| c | c |}
	\hline
	Arvu $a$ pöördarv & $\dfrac{1}{a}$ \\
	\hline
	Arvu $a$ vastandarv & $-a$ \\
	\hline
	Arvu $a$ ja $b$ summa & $a+b$ \\
	\hline
	Arvu $a$ ja $b$ vahe & $a-b$ \\
	\hline
\end{tabular}

\vspace{5mm}
\hspace{5mm}
\textbf{\underline{Järjestikused arvud}}

\vspace{2mm}
\hspace{5mm}
Vahest esineb ka ülesandeid, mis nõuavad leida järjestikuseid arve.\\
\hspace{5mm}
Selliste ülesannete puhul tuleb eeldada, et arv millest järjestikuste arvude loendamist alustatakse on\\ \hspace{5mm} tundmatu ehk \textbf{x}. On olemas erinevaid sorte järjestikuseid arve, põhilsed neist on: \\
\vspace{2mm}
\hspace{5mm}
1) Järjestikused  (täis)arvud - arvud, mis kasvavad või kahanevad 1 arvu võrra.\\
\vspace{2mm}
\hspace{10mm}
Üldine kuju: $x$, $x+1$, $x+2$, $x+3$ jne...\\
\vspace{2mm}
\hspace{5mm}
2) Järjestikused paarisarvud - arvud, mille järjestikune jada kirjeldab paarisarve (ehk, arvud\\ \hspace{5mm} kasvavad/kahanevad sammuga 2).\\
\vspace{2mm}
\hspace{10mm}
Üldine kuju: $x$, $x+2$, $x+4$, $x+6$ jne...\\
\vspace{2mm}
\hspace{5mm}
3) Järjestikused paaritud arvud - arvud, mille järjestikune jada kirjeldab paarituid arve(ehk, arvud\\ \hspace{5mm} kasvavad/kahanevad sammuga 2).\\
\vspace{2mm}
\hspace{10mm}
Üldine kuju: $x$, $x+2$, $x+4$, $x+6$ jne...\\
\vspace{2mm}
\hspace{5mm}
4) Järjestikused arvud fikseeritud sammuga - säärased arvud, kasvavad/kahanevad kindla määratud\\ \hspace{5mm} sammuga (nt sammuga 7 või 0.123)\\

\vspace{2mm}
\hspace{5mm}
Nagu näha, paarisarvude ja paaritute arvude jadad üldisel kujul tegelikult ei erinegi. See on sellepärast,\\ \hspace{5mm} et mõlemal juhul on üleminek esimeselt arvult teisele sammuga kaks. Ehk kui jada esimeseks arvuks\\ \hspace{5mm} ($x$-iks) oleks  paaritu arv 5, siis iga kahe sammu tagant esineksid meil samuti paaritud arvud ehk: 5, 7,\\ \hspace{5mm} 9, 11 jne...\\
\hspace{5mm}
Sama lugu ka paarisarvudega. Kui jada esimeseks arvuks oleks paarisarv 6, siis iga kahe sammu tagant\\ \hspace{5mm} esineksid samuti paarisarvud: 6, 8, 10, 12, 14 jne...\\

\vspace{5mm}
\hspace{5mm}
\textbf{Näited}\\
\vspace{2mm}
\hspace{5mm}
1) Arvust x viie võrra suurem arv. \hspace{5mm} Vastus: $x+5$\\
\hspace{5mm}
2) Arvust x nelja võrra väiksem arv. \hspace{5mm} Vastus: $x-4$\\
\hspace{5mm}
3) Arvust x neli korda suurem arv. \hspace{5mm} Vastus: $4\cdot x$\\ 
\vspace{2mm}
\hspace{7mm}
jätkub järgmisel lehel...
\end{flushleft}
}}}
\end{center}

\newpage

\begin{center}
\fbox{\fbox{\parbox{6.5in}{\centering
\begin{flushleft}




\hspace{5mm}
4) Arvust x seitse korda väiksem arv. \hspace{5mm} Vastus: $\dfrac{x}{7}$\\
\hspace{5mm}
5) Arvu 19 pöördarv. \hspace{5mm} Vastus: $\dfrac{1}{19}$\\
\hspace{5mm}
6) Arvu -38 vastandarv. \hspace{5mm} Vastus: $-(-38)=38$\\
\hspace{5mm}
7) Arvu 4 ja 8 summa. \hspace{5mm} Vastus: $4+8$\\
\hspace{5mm}
8) Arvu 7 ja 5 vahe. \hspace{5mm} Vastus: $7-5$





\vspace{5mm}
\hspace{5mm}
\textbf{Keerulisem näide}\\
\vspace{2mm}
\hspace{5mm}
On antud 4 järjestikust paaritut arvu. Kui kõige väiksemast antud arvust lahutada ülejäänud 3 arvu,\\ \hspace{5mm} saame tulemuseks $-22$. Leia need arvu.\\
\vspace{5mm}
\hspace{5mm}
Olgu esimene arv $x$.\\
\hspace{5mm}
Siis teine arv on 2 võrra suurem, ehk $x+2$.\\ 
\hspace{5mm}
Kolmas on eelmisest 2 võrra suurem, ehk $(x+2)+2=x+4$.\\
\hspace{5mm}
Ja viimaks neljas: $x+6$.

\vspace{2mm}
\hspace{5mm}
Lahutame esimesest arvust ülejäänud kolm arvu.\\
\vspace{2mm}
\hspace{5mm}
Saame järgmise avaldise: $x-(x+2)-(x+4)-(x+6)=-22$\\
\vspace{2mm}
\hspace{5mm}
Lihtsustame võrrandit, avades ennekõike sulud,  siis liidame sarnased liikmed kokku ning viime\\ \hspace{5mm} tundmatud ühele poole ja arvud teisele poole võrdusmärki.

\begin{equation}
\label{eq14_1}
x-x-2-x-4-x-6=-22
\end{equation}

\begin{equation}
\label{eq14_2}
x-x-x-x=-22+2+4+6
\end{equation}

\begin{equation}
\label{eq14_3}
-2x=-10
\end{equation}

\begin{equation}
\label{eq14_4}
-2x=-10 \bigg| : -2
\end{equation}

\begin{equation}
\label{eq14_5}
\dfrac{-2x}{-2}=\dfrac{-10}{-2}
\end{equation}

\begin{equation}
\label{eq14_6}
x=5
\end{equation}\\
\vspace{2mm}
\hspace{5mm}
Jadas esimene paaritu arv on 5. Järelikult meie 4 järjestikust paaritut arvu on: 5, 7, 9, 11.
\end{flushleft}
}}}
\end{center}


\vspace{0.5cm}

\textbf{Märkmed}\\
\vspace{2mm}
\begin{mdframed}[style=graphpaper]
\vspace{3cm}
\end{mdframed}
