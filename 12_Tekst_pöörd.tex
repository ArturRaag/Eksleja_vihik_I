\begin{center}
\fbox{\fbox{\parbox{6.5in}{\centering
\begin{flushleft}

\hspace{5mm}
\textbf{\underline{Pöördvõrdelise seose ülesanded}}

\vspace{5mm}
\hspace{5mm} Oletame, et 2 meest saavad enda töö tehtud 16 tunniga. Kui kiiresti saaksid sama töö tehtud 8\\ \hspace{5mm} meest? Ehk...\\

\begin{equation}
2 \text{ meest} \longrightarrow 16 \text{ tundi}
\end{equation}

\begin{equation}
8 \text{ meest} \longrightarrow x \text{ tundi}
\end{equation}

\vspace{2mm}
\hspace{5mm} Kui me proovime lahendada ülesannet samamoodi, nagu eelmises peatükkis, siis leiaksime, et saame\\ \hspace{5mm} naljaka tulemuse.

\begin{equation}
\label{12_3}
\dfrac{2}{8}=\dfrac{16}{x} \hspace{2mm} \longrightarrow \hspace{2mm} x = \dfrac{16 \cdot 8}{2} = 64
\end{equation}

\hspace{5mm} Saime, et 8 mehega võtaks töö tegemine 64 tundi. Kuid 2 mehega kulus meil töö tegemiseks 16 tundi?\\ \hspace{5mm} Selline järeldus ei kõlba kuhugi.\\
\vspace{2mm}
\hspace{5mm} Õige tulemuse leidmiseks, tuleb vaid üks võrduse pooltest kirjutada tagurpidi (miks?). Kirjutame \\ \hspace{5mm} tagurpidi vaid võrduse parema poole. Saame:

\vspace{2mm}
\hspace{5mm}  
\begin{equation}
\label{12_4}
\dfrac{2}{8}=\dfrac{x}{16} \hspace{2mm} \longrightarrow \hspace{2mm} x = \dfrac{2 \cdot 16}{8} = 4
\end{equation}

\hspace{5mm} Järelikult kulub 8-l mehel sama töö tegemiseks vaid 4 tundi.\\

\vspace{5mm}
\hspace{5mm} \textbf{\underline{Alternatiivne viis pöördvõrdelise seose tesktülesannete lahendamiseks}}

\vspace{5mm}
\hspace{5mm} Märkimisväärselt mugavam on aga kasutada pöördvõrdelise seose omadust, mis ütleb, et suuruste\\ \hspace{5mm} korrutis on konstantne ehk muutumatu.

\vspace{2mm}
\hspace{5mm} Ehk meie näite puhul: 

\begin{equation}
\label{12_5}
2\text{ meest}\cdot 16 \text{ tundi} = 8 \text{ meest} \cdot x \text{ tundi}
\end{equation}

\vspace{2mm}
\hspace{5mm} millest saame

\begin{equation}
\label{12_6}
2\cdot 16 = x \cdot 8 \bigg| : 8
\end{equation}

\begin{equation}
\label{12_7}
\dfrac{2\cdot 16}{8} = \dfrac{x\cdot \cancel{8}}{\cancel{8}} \hspace{2mm} \longrightarrow \hspace{2mm} x = 4 \text{(tundi)}
\end{equation}

\vspace{2mm}
\hspace{5mm} Sisuliselt võib avalidsest \ref{12_5} mõelda samuti kui "kaalu süsteemist". Siin on vaid tasakaalus asjaolu, et\\ \hspace{5mm} vähese jõuga tehtud töö võtab rohkem aega, kuid kiirema aja saavutamiseks on vaja omakorda \\ \hspace{5mm} rohkem jõudu. Ehk ühe suuruse suurenemisel (näiteks meeste arvu suurenemisel), peab teine suurus\\ \hspace{5mm} vähenema (näiteks töö aeg). Kui aga vähendada üheaegselt mõlemat suurust (meeste arvu JA aega),\\ \hspace{5mm} siis ei oleks esiteks selline nähtus mõistuse poolest väga loomulik ning matemaatiliselt lükkaks see\\ \hspace{5mm} meil võrduse tasakaalust välja.
\end{flushleft}
}}}
\end{center}


\newpage

\vspace{1cm}

\textbf{Märkmed}\\
\vspace{2mm}
\begin{mdframed}[style=graphpaper]
\vspace{21cm}
\end{mdframed}