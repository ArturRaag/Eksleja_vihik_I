\fbox{\fbox{\parbox{165mm}{\centering

\begin{flushleft}
\vspace{5mm}
\hspace{5mm} \textbf{Murd kui jagamistehe:} $\dfrac{lugeja}{nimetaja}=l:n$ \hspace{3mm} $n \neq 0$, 
\hspace{5mm} Näiteks: $\dfrac{3}{4}=3:4=0.75$,\\  \vspace{5mm}
\hspace{5mm} \textbf{Laiendamine:} ${\dfrac{a}{b}}^{(n}= \dfrac{a\cdot n}{b\cdot n}$, \hspace{44mm} Näiteks: ${\dfrac{3}{7}}^{(4}= \dfrac{3\cdot 4}{7\cdot 4}=\dfrac{12}{28}$, \\
\vspace{5mm}
\hspace{5mm} \textbf{Täisarv murruks:} $a=\dfrac{a}{1}$, \hspace{45.5mm} Näiteks: $ 13=\dfrac{13}{1}$, \\
\vspace{5mm}
\hspace{5mm} \textbf{Taandamine:} $\dfrac{a\cdot \cancel{n}}{b\cdot \cancel{n}}=\dfrac{a}{b}$, \hspace{48mm} Näited: a) $\dfrac{15}{35}=\dfrac{3 \cdot \cancel{5}}{7 \cdot \cancel{5}}=\dfrac{3}{7}$\\ 
\vspace{5mm} \hspace{107mm} b) $\dfrac{12}{26}=\dfrac{6 \cdot \cancel{2}}{13 \cdot \cancel{2}}=\dfrac{6}{13}$ \\
\vspace{10mm}
\hspace{5mm} \textbf{Segaarv liigmurruks:} $A\dfrac{b}{c}= \dfrac{b+A\cdot c }{c}$, 
\hspace{26mm} Näiteks: $2\dfrac{3}{5}=\dfrac{3+2\cdot 5}{5}=\dfrac{13}{5}$, \\
\vspace{5mm}
\hspace{5mm}
\textbf{Nimetajas ja lugejas ühesugune arv:} $\dfrac{a}{a}=1$, \hspace{3mm} $a \neq 0$ \\ \vspace{5mm}
\hspace{5mm}
Näiteks: a) $\dfrac{4}{4}=1$,\\
\vspace{5mm}
\hspace{19mm} b) $\dfrac{35}{35}=1$\\

\vspace{5mm}
\hspace{5mm} \textbf{Kümnendmurd murruks:} a) $0.75 = \dfrac{75}{100}= \dfrac{3\cdot \cancel{25}}{4\cdot \cancel{25}}=\dfrac{3}{4}$\\
\vspace{5mm}
\hspace{50.5mm} b) $0.0125= \dfrac{125}{10 000}=\dfrac{1 \cdot \cancel{125}}{80 \cdot \cancel{125}}=\dfrac{1}{80}$\\
\vspace{5mm}
\hspace{50.5mm} c) $3.05=\dfrac{305}{100}=\dfrac{61 \cdot \cancel{5}}{20 \cdot \cancel{5}}=\dfrac{61}{20}=3\dfrac{1}{20}$

\end{flushleft}
}}}

\newpage

\textbf{Autori poolsed kommentaarid}\\

1) Igat täisarvu saab teisendada murruks: $a=\dfrac{a}{1}$. Kuna mistahes arv $a$ jagatud $1$-ga annab meile selle sama esialgse arvu $a$. (Ehk lühidalt: $1-ga$ jagades meil miski ei muutu.)\\

2) Kui ei ole kindel, milline on kõige suurem arv, millega murd taandub, siis alusta $2-st$. Kui $2-ga$ enam ei taandu, siis suurenda arvu järku 1 võrra, ehk proovi taandada $3-ga$. Kui kolmega enam ei taanud, siis proovi $4-ga$ jne. Arv millega taandad ei tohiks ületada murru nimetajat ega lugejat.\\

3) Kui teisendad kümnendmurdu (komaga arvu) murruks, siis kirjuta lugejasse KÕIK kümnendmurrus esinevad arvud ilma komata, seejuures pidage silmas, et kui kümnendmurd algab nullidega, siis neid kirjutada vaja ei ole. Seejärel kirjuta nimetajasse number "$1$" millele järgneb nii palju nulle, kui palju on komakohti kümnendmurrul. (vaata kastis olevaid näiteid)\\
\vspace{1.5cm}

\textbf{Märkmed}\\
\vspace{2mm}
\begin{mdframed}[style=graphpaper]
\vspace{14cm}
\end{mdframed}
