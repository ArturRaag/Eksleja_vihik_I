\begin{center}
\fbox{\fbox{\parbox{6.5in}{\centering
\begin{flushleft}


\hspace{5mm} \textbf{\underline{Tähtavaldiste lihtsustamine}}\\
\vspace{5mm}
\hspace{5mm} Avaldised kujul $3x+4y-7x^{2}-13x$ omavad erinevaid liikmeid. Liikmete all mõeldakse erinevaid\\ \hspace{5mm} "number-täht" paare. Kusjuures, kui täht on ka mingis astmes, siis peetakse seda eraldi liikmeks.\\
\vspace{2mm}
\hspace{5mm} \textbf{Näiteks:} $3x$, $7x^{2}$ ja $5y$ on kõik \underline{erinevad liikmed}, küll aga $3x$ ja $13x$ on \underline{sarnased liikmed}.\\
\vspace{5mm}
\hspace{5mm} Kui tähe ees numbrit ei ole, siis vaikimisi on tal ees number "1".\\
\vspace{2mm}
\hspace{5mm} \textbf{Näiteks:} $ x^{2} = 1 \cdot x^{2}$ (Need on samad asjad).

\vspace{5mm}
\hspace{5mm} Sarnaseid liikmeid saab kokku liita või lahutada (nimetatakse koondamiseks).\\
\vspace{2mm}
\hspace{5mm} \textbf{Näiteks:} $ 3x+5y+2x-2y = (3+2)x+(5-2)y = 5x+3y $

\vspace{5mm}
\hspace{5mm} Koondamisel kehtib kaks lihtsat reeglit: 1) liidetavate liikmete järjekorda võib muuta.\\
\vspace{2mm}
\hspace{67mm} 2) liidetavaid saab rühmitada.\\
\vspace{5mm}
\hspace{5mm} Juhul, kui avaldis on sulgudes ja nende ees on mõni kordaja, siis korrutatakse kõik sulgudes olevad\\ \hspace{5mm} liikmed sulu ees oleva arvuga.\\

\vspace{2mm}
\hspace{5mm} \textbf{Näiteks:} Lihtsustame järgmise avaldise: \[ 3(x+4y)-2x(2+z)= \]

\[ 3 \cdot x + 3 \cdot 4y - 2x \cdot 2 - 2x \cdot z= \]

 \[ 3x + 12y - 4x -2xz \]

\hspace{5mm} Kuna $3x$ ja $-4x$ on sarnased, saame need kokku liita: ($3x-4x=-1x=-x$)

\hspace{5mm}
Seega saame esialgse avaldise lihtsustamisel tulemuseks avaldise

\[ -x + 12y-2xz\]

\vspace{5mm}
\hspace{5mm} \textbf{\underline{Võrrandite põhiomadused}}\\
\vspace{5mm}
\hspace{5mm} Olgu meil võrrand

\begin{equation}
\label{eq13_1}
26 = \dfrac{3x}{5}+2
\end{equation}

\vspace{2mm}
\hspace{5mm} Proovime leida selle võrrandi lahendi ehk $x$-i väärtuse. Esmalt tasuks teha märkus, et võrrand \ref{eq13_1} on\\ \hspace{5mm} sama mis \ref{eq13_2}. Sellest järeldub, esimene omadus: \textbf{võrrandi pooli võib julgelt vahetada}.

\begin{equation}
\label{eq13_2}
\dfrac{3x}{5}+2 = 26
\end{equation}

\hspace{5mm} Näide jätkub järgmisel lehel...


\end{flushleft}
}}}
\end{center}



\newpage

\begin{center}
\fbox{\fbox{\parbox{6.5in}{\centering
\begin{flushleft}

\hspace{5mm} Kuna eesmärk on leida x (ehk üritame saada kujule $x=...$), siis tuleks meil kuidagi nendest\\ \hspace{5mm} üleliigsetest arvudest lahti saada. \\
\vspace{2mm}
\hspace{5mm} Ettepanek oleks mõelda võrrandist kui kaalust. Me võime kaalu ühelt poolt midagi ära võtta või\\ \hspace{5mm} juurde lisada, kuid siis langeks kaal tasakaalust välja. Kui aga võtta (või juurde lisada) sama kogus\\ \hspace{5mm} kaalu MÕLEMALT poolt, siis säilitab kaal oma tasakaalu ja midagi ümber ei kuku.\textbf{ Eesmärk olekski \\ \hspace{5mm} säilitada igavesti tasakaalu}.\\
\vspace{2mm}
\hspace{5mm} Alustame sellest, et eemaldame võrrandi paremalt poolt üksildase 2.\\
\vspace{2mm}
\hspace{5mm} Nagu eelnevalt mainitud, tuleb meil selleks lahutada võrrandi mõlemalt poolt 2, kuna $2-2=0$.\\ \hspace{5mm} (piltlikult öeldes, eemaldame kaalu mõlemalt poolt 2kg, või 2g)

\begin{equation}
\label{eq13_3}
26 = \dfrac{3x}{5}+2 \hspace{5mm} \bigg| -2
\end{equation}

\begin{equation}
\label{eq13_4}
26-2 = \dfrac{3x}{5}+2-2
\end{equation}

\begin{equation}
\label{eq13_5}
26-2 = \dfrac{3x}{5}
\end{equation}

\hspace{5mm}
Oleme sammu võrra lähemal. Nüüd võiks proovida lahti saada nimetajas olevast 5st. Meenutades\\ \hspace{5mm} murdude omadusi, teame, et kui murd on kujul $\dfrac{a}{1}$, siis see on sama, mis lihtsalt $a$, kuna 1-ga jagamine\\ \hspace{5mm} meil midagi ei muuda. Ehk nõnda saaksime me nimetajast täiesti lahti. Kuid kuidas nimetaja 1-ks\\ \hspace{5mm} saada? Jällegi, meenutades murdude omadusi, meenub meil, et kui meil on murru lugejas ja nimetajas\\ \hspace{5mm} ühesugused arvud, siis need arvud taanduvad 1-ks. Ehk nt $\dfrac{x \cdot \cancel{a}}{\cancel{a}}=\dfrac{x}{1}=x$.\\
\vspace{2mm}
\hspace{5mm} Järelikult mida me saaksime teha, on korrutada võrduse \ref{eq13_5} paremat poolt 5-ga läbi, siis taanduksid\\ \hspace{5mm} meil 5-ed ära. Kuid siin kerkib probleem. Kui me korrutame vaid võrduse paremat poolt 5-ga, siis\\ \hspace{5mm} piltlikult öeldes, teeme me vaid kaalu parema poole 5 korda raskemaks, ignoreerides kaalu vasakut\\ \hspace{5mm} poolt. Teisisõnu, meil kukkub kaal ümber. Et ümberkukkumist vältida, tuleb ka vasak pool 5-ga läbi \\ \hspace{5mm} korrutada.\\

\begin{equation}
\label{eq13_6}
24 = \dfrac{3x}{5} \hspace{2mm} \bigg| \cdot 5 
\end{equation}


\begin{equation}
\label{eq13_7}
24 \cdot 5 = \dfrac{3x \cdot 5}{5} 
\end{equation}

\begin{equation}
\label{eq13_8}
24 \cdot 5 = \dfrac{3x \cdot \cancel{5}}{\cancel{5}} 
\end{equation}

\begin{equation}
\label{eq13_9}
24 \cdot 5 = \dfrac{3x \cdot 1}{1} 
\end{equation}

\begin{equation}
\label{eq13_10}
24 \cdot 5 = 3x
\end{equation}

\hspace{5mm}
Viimaks, kasutades samasugust loogikat, vabaneme x-i ees olevast 3-st, kuid see kord mitte korrutades\\ \hspace{5mm} 3-ga, vaid ikka jagades.

\begin{equation}
\label{eq13_11}
24 \cdot 5 = 3x \hspace{2mm} \bigg| : 3
\end{equation}

\hspace{5mm} Näide jätkub järgmisel lehel...

\end{flushleft}
}}}
\end{center}



\begin{center}
\fbox{\fbox{\parbox{6.5in}{\centering
\begin{flushleft}


\begin{equation}
\label{eq13_12}
\dfrac{24 \cdot 5}{3} = \dfrac{3x}{3}
\end{equation}

\begin{equation}
\label{eq13_13}
\dfrac{24 \cdot 5}{3} = \dfrac{\cancel{3}\cdot x}{\cancel{3}}
\end{equation}


\begin{equation}
\label{eq13_14}
\dfrac{24 \cdot 5}{3} = \dfrac{1\cdot x}{1}
\end{equation}

\begin{equation}
\label{eq13_15}
\dfrac{24 \cdot 5}{3} = x
\end{equation}

\begin{equation}
\label{eq13_16}
40 = x
\end{equation}

\hspace{5mm} Ja ongi võrrandi lahend leitud, x=40.

\vspace{5mm}
\hspace{5mm}
Võrrandit lahendades, jõudsime järgmiste järeldusteni:\\
\vspace{2mm}
\hspace{5mm}
1) Võrrandi pooli võib vahetada. (Järeldus avaldisest \ref{eq13_1} ja \ref{eq13_2}.)

\hspace{5mm}
2) Võrrandi mõlemale poolele võib liita või lahutada sama arvu. (Järeldus avaldisest \ref{eq13_3} kuni \ref{eq13_5})

\hspace{5mm}
3) Võrrandi mõlemat poolt, võib korrutada või jagada ühe ja sama arvuga (välja arvatud nulliga!)\\
\hspace{7mm} (Järeldus avaldisest \ref{eq13_6} - \ref{eq13_10} ja \ref{eq13_11} - \ref{eq13_16})


\vspace{5mm}
\hspace{5mm} \textbf{\underline{Mõned tähelepanekud veel}}

\vspace{2mm}
\hspace{5mm} Kui uurida lähemalt avaldisi \ref{eq13_3} ja \ref{eq13_5}, siis on märgata, et tegelikult tõstsime me ju selle 2 lõpuks\\ \hspace{5mm} lihtsalt teisele poole võrdust ja muutsime ka selle märki. Ehk tegelikult, kui me oleks seda varem\\ \hspace{5mm} teadnud, siis oleks võinud avaldise \ref{eq13_4} üldsegi vahele jätta.\\
\vspace{2mm}
\hspace{5mm} Siit saab teha veel ühe järelduse:\\

\hspace{5mm} \textbf{Võrrandi liikmeid (vaid need, mis on eristatud kas + või - märgiga) võib viia ühelt poolt\\ \hspace{5mm} võrdusmärki teisele, muutes kindlasti ka liikme ees olevat märki!}

\vspace{2mm}
\hspace{5mm} \textbf{Näiteks:} $3x-5=x+10 \hspace{2mm} \longrightarrow \hspace{2mm} 3x-x=10+5$\\

\vspace{5mm}
\hspace{5mm} \textbf{\underline{Ristkorrutis}}

\vspace{2mm}
\hspace{5mm} 
Kui uurida täpsemalt avaldist \ref{eq13_6} ja võrrelda seda avaldisega \ref{eq13_15}, siis on näha, et ka\\ \hspace{5mm} siin liikusid meie arvud paremalt poolt võrdusmärki vasakule poole, kuid natuke teisiti. Nimetajas\\ \hspace{5mm} olev 5 jõudis meil vasakule poole lugejasse ning paremal pool lugejas olev 3 liikus meil vasakule poole\\ \hspace{5mm} nimetajasse. Ehk lühidalt, \textbf{arvud liikusid meil risti teisele poole võrdust!}\\
\vspace{2mm}
\hspace{5mm} Oleks me ka seda varem teadnud, siis oleks me võinud sammud \ref{eq13_7} - \ref{eq13_14} vahele jätta. 

\vspace{5mm}
\hspace{5mm} \textbf{Näiteks:} viime ristkorrutise harjutamise eesmärgil tähed vasakule, arvud paremale:

\begin{equation}
\label{eq13_17}
\dfrac{2a}{3b}=\dfrac{8y}{9x}
\end{equation}

\begin{equation}
\label{eq13_18}
\dfrac{a \cdot x}{b \cdot y}=\dfrac{8\cdot 3}{9\cdot 2}
\end{equation}

\begin{equation}
\label{eq13_19}
\dfrac{a x}{by}=\dfrac{24}{18}
\end{equation}



\end{flushleft}
}}}
\end{center}


\newpage
\vspace{1cm}

\textbf{Märkmed}\\
\vspace{2mm}
\begin{mdframed}[style=graphpaper]
\vspace{19cm}
\end{mdframed}