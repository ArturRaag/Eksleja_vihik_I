\begin{center}

\fbox{\fbox{\parbox{6.5in}{\centering
\begin{flushleft}
\vspace{2mm}
\hspace{5mm} \textbf{\underline{Osa leidmine arvust:}}\\
\vspace{2mm}
\hspace{5mm} $\dfrac{a}{b}$ arvust $C = \dfrac{a}{b}\cdot C = \dfrac{a}{b}\cdot \dfrac{C}{1} = \dfrac{a \cdot C}{b \cdot 1} = \dfrac{a\cdot C}{b}$ \\
\vspace{2mm}
\hspace{5mm} \textbf{Näiteks:} $\dfrac{2}{5}$ arvust $10 = \dfrac{2}{5} \cdot 10 = \dfrac{2}{5} \cdot \dfrac{10}{1} = \dfrac{2 \cdot 10}{5 \cdot 1} = \dfrac{20}{5} =$ \Big[ Taandame 5-ga\Big] $= \dfrac{4}{1} = 4:1 = 4$\\
\vspace{5mm}
\hspace{5mm} \textbf{\underline{Protsent kui murd:}} \fbox{$A\%=\dfrac{A}{100}$}\\
\vspace{2mm}
\hspace{5mm} \textbf{Näiteks:} $15\%=\dfrac{15}{100}=\dfrac{3}{20}$\\
\vspace{5mm}
\hspace{5mm} \textbf{\underline{Protsent arvust:}}\\
\vspace{2mm}
\hspace{5mm} $A\%$ arvust $B = \dfrac{A}{100}$ arvust $B = \dfrac{A}{100} \cdot \dfrac{B}{1} = \dfrac{A\cdot B}{100 \cdot 1}= \fbox{$\dfrac{A \cdot B}{100}$}$\\
\vspace{2mm}
\hspace{5mm} \textbf{Näiteks:} $25\%$ arvust $ 400 = \dfrac{25}{100} \cdot 400 = \dfrac{25}{100} \cdot \dfrac{400}{1} = \dfrac{25 \cdot 400}{100 \cdot 1} =$ \Bigg[\begin{tabular}{c}
Siin on näha,\\
et saame taandada\\
100-ga\end{tabular}\Bigg] $=$\\
\vspace{2mm}
\hspace{5mm} $= \dfrac{25 \cdot \cancel{400}}{\cancel{100} \cdot 1} = \dfrac{25 \cdot 4}{1 \cdot 1} = \dfrac{100}{1}=100:1=100$

\vspace{5mm}
\hspace{5mm}
\textbf{\underline{Terviku leidmine}}

\vspace{2mm}
\hspace{5mm}
Leia arv $X$, millest $A\%$ on $B$

\vspace{2mm}
\hspace{5mm}
$B:\dfrac{A}{100} = X \longrightarrow \dfrac{B}{1}\cdot \dfrac{100}{A} = X \longrightarrow \dfrac{B \cdot 100}{1\cdot A} = X \longrightarrow \fbox{$\dfrac{100 \cdot B}{A} = X$}$

\vspace{5mm}
\hspace{5mm}
\textbf{Näiteks:} Kui $12\%$ arvust $X$ on $20$, siis

\vspace{5mm}
\hspace{5mm}
$X= 20:\dfrac{12}{100}=\dfrac{20 \cdot 100}{12}= \dfrac{2000}{12}\approx 166.67$

\vspace{5mm}
\hspace{5mm}
\textbf{\underline{Protsendi leidmine}}

\vspace{2mm}
\hspace{5mm}
Selleks, et leida mitu protsenti on arv $b$ arvust $a$ suurem või väiksem: 
\fbox{$p\%=\dfrac{|a-b|}{a}\cdot 100\%$}

\vspace{2mm}
\hspace{5mm}
\textbf{Näiteks:}

\vspace{2mm}
\hspace{5mm}
Õpilane oli aasta alguses 130 cm pikk. Aasta lõpuks oli õpilane 150 cm pikkune. Kui suur oli\\ \hspace{5mm} juurdekasv protsentides?

\vspace{2mm}
\hspace{5mm}
$\dfrac{|130-150|}{130} \cdot 100\% =\dfrac{20}{130}\cdot 100\% \approx 15.38\% $

\vspace{2mm}
\hspace{5mm}
Järelikult õpilase pikkus suurenes aasta algusega võrreldes ligikaudu $15.38\%$ võrra.

\end{flushleft}
}}}
\end{center}

\newpage

\begin{center}

\fbox{\fbox{\parbox{6.5in}{\centering
\begin{flushleft}

\vspace{2mm}
\hspace{5mm}
\textbf{\underline{Protsendi võrra suurendamine/vähendamine}}

\vspace{2mm}
\hspace{5mm}
\textbf{Üldine valem}

\vspace{2mm}
\hspace{5mm}
\fbox{$b=a\cdot \left(1 \pm \dfrac{p}{100} \right) = a \cdot (1 \pm 0.01 \cdot p)$}

\vspace{2mm}
\hspace{5mm}
kus $b$ - uus suurus, $a$ - algne suurus, $p$ - protsent täisarvuna.

\vspace{2mm}
\hspace{5mm}
Märgi $\pm$ kasutusest:\\
\vspace{2mm}
\hspace{5mm}
1) Kui me peame suurust $a$ \textbf{SUURENDAMA} $p\%$ võrra, siis kasutame ''+'' märki.

\hspace{5mm}
2) Kui me peame suurust $a$ \textbf{VÄHENDAMA} $p\%$ võrra, siis kasutame ''$-$'' märki.


\vspace{5mm}
\hspace{5mm}
\textbf{Näide:}

\vspace{2mm}
\hspace{5mm}
Arvu $45$ vähendamisel $13\%$ võrra saame:

\hspace{5mm}
\begin{equation}
\label{eq3_1}
45 \cdot \left( 1-\dfrac{13}{100} \right)= 39.15
\end{equation}

\vspace{2mm}
\hspace{5mm}
NB! Kui mingit arvu vähendatakse kindla protsendi võrra ning seejärel suurendatakse \textbf{sama} protsendi\\ \hspace{5mm} võrra, siis esialgse arvuni me tagasi välja ei jõua! Sama kehtib ka vastupidisel juhul.\\ \hspace{5mm} Demonstreerime seda, suurendades eelmist vastust (\ref{eq3_1}) $13\%$ võrra:

\vspace{2mm}
\hspace{5mm}
\begin{equation}
\label{eq3_2}
39.15 \cdot \left( 1+\dfrac{13}{100} \right) = 44.2395 \neq 45
\end{equation}


\end{flushleft}
}}}
\end{center}

\textbf{Märkmed}\\
\vspace{2mm}
\begin{mdframed}[style=graphpaper]
\vspace{10cm}
\end{mdframed}