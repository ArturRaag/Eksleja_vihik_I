\begin{center}
\fbox{\fbox{\parbox{6.5in}{\centering
\begin{flushleft}


\vspace{2mm}
\hspace{5mm}
\textbf{\underline{Ruutjuur}}

\vspace{2mm}
\hspace{5mm}
Astendamise pöördtehet nimetatakse \textbf{juurimiseks}.

\vspace{2mm}
\hspace{5mm}
Ehk kui astendamise puhul on meil näiteks tehe $12^{2}=12 \cdot 12 = 144$, siis selle pöördtehe (vastupidine\\ \hspace{5mm} tehe) näeks välja nõnda: $\sqrt{144}=12$.

\vspace{2mm}
\hspace{5mm}
Teisisõnu ruutjuure puhul me küsime, mis oli see arv, mis võeti ruutu (astmesse $2$) mis andis käesoleva\\ \hspace{5mm} arvu. On ka võimalik leida muudest astmetest pärinevaid arve, näiteks kuubist: $12^{3}=12 \cdot 12 \cdot 12 = 1728$. \\ \hspace{5mm} Sellisel juhul, oleks meil tegemist kuupjuurega ja juure leidmine näeks välja nii: $\sqrt[3]{1728}=12$.

\vspace{2mm}
\hspace{5mm}
Juurimise üldine kuju:

\begin{equation}
\sqrt[n]{b}=a
\end{equation}

\vspace{2mm}
\hspace{5mm}
kus $n$ - juurija, $b$ - juuritav, $a$ - juur, ning $\sqrt{\phantom{x}}$ on juuremärk.

\vspace{2mm}
\hspace{5mm}
Selles vihikus piirdume vaid ruutjuurtega, kusjuures juurija $n$ võib ruutjuure puhul kirjutamata jätta.\\ \hspace{5mm} \textbf{Ruutjuurt tohib võtta AINULT positiivsest arvust}.

\vspace{2mm}
\hspace{5mm}
\textbf{Näited:}

\vspace{2mm}
\hspace{5mm}
1) $\sqrt{16}=4$, kuna $4^{2}=16$ \hspace{20mm} 2) $\sqrt{0.25}=0.5$, kuna $0.5^{2}=0.25$

\vspace{2mm}
\hspace{5mm}
3) $\sqrt{-25}$ puudub, kuna ei leidu ühtegi arvu, mille puhul $n^{2}=-25$ 

\vspace{5mm}
\hspace{5mm}
\textbf{\underline{Tehted juurtega}}

\vspace{5mm}
\hspace{5mm}
\textbf{Ruutjuur ruudust}
\begin{equation}
\label{35_eq1}
\boxed{\sqrt{a^{2}}=|a|} 
\end{equation}

\hspace{5mm} Näited:

\vspace{2mm}
\hspace{5mm}
1) $\sqrt{12^{2}}=12$

\vspace{2mm}
\hspace{5mm}
2) $\sqrt{(-9)^{2}}=9$

\vspace{5mm}
\hspace{5mm}
\textbf{Korrutise ruutjuur}

\vspace{2mm}
\begin{equation}
\label{35_eq2}
\boxed{\sqrt{ab}=\sqrt{a} \cdot \sqrt{b}} \hspace{2mm} a \geq 0, \hspace{2mm} b \geq 0
\end{equation}

\hspace{5mm}
Näited:

\vspace{2mm}
\hspace{5mm}
1) $\sqrt{225}=\sqrt{9 \cdot 25}=\sqrt{9} \cdot \sqrt{25}=3 \cdot 5= 15$

\vspace{2mm}
\hspace{5mm}
2) $\sqrt{2}\cdot \sqrt{8}=\sqrt{2\cdot 8}=\sqrt{16}=4$

\vspace{5mm}
\hspace{5mm}
\textbf{Jagatise ruutjuur}

\begin{equation}
\label{35_eq3}
\boxed{\sqrt{\dfrac{a}{b}}=\dfrac{\sqrt{a}}{\sqrt{b}}} \hspace{2mm} a \geq 0, \hspace{2mm} b > 0
\end{equation}




\end{flushleft}
}}}
\end{center}

\pagebreak
\begin{center}
\fbox{\fbox{\parbox{6.5in}{\centering
\begin{flushleft}

\vspace{2mm}
\hspace{5mm}
Näited:

\vspace{2mm}
\hspace{5mm}
1) $\sqrt{\dfrac{25}{9}}=\dfrac{\sqrt{25}}{\sqrt{9}}=\dfrac{5}{3}$

\vspace{2mm}
\hspace{5mm}
2) $\dfrac{\sqrt{6}}{\sqrt{24}}=\sqrt{\dfrac{6}{24}}=\sqrt{\dfrac{1}{4}}=\dfrac{\sqrt{1}}{\sqrt{4}}=\dfrac{1}{2}$

\vspace{5mm}
\hspace{5mm}
\textbf{Astme ruutjuur}

\begin{equation}
\label{35_eq4}
\boxed{\sqrt{a^{n}}=\left( \sqrt{a} \right)^{n}} \hspace{2mm} a \geq 0
\end{equation}

\vspace{2mm}
\hspace{5mm}
Näiteks:

\vspace{2mm}
\hspace{5mm}
$\sqrt{16^{3}}=(\sqrt{16})^{3}=4^{3}=64$

\vspace{5mm}
\hspace{5mm}
\textbf{Tegurdamine}

\begin{equation}
\label{35_eq5}
\boxed{\sqrt{k^{2}\cdot a}=k\sqrt{a}}
\end{equation}

\hspace{5mm}
NB! Tasub rõhutada, et kõik valemid ja teisendused kehtivad mõlemat pidi.

\vspace{2mm}
\hspace{5mm}
Näiteks:

\vspace{2mm}
\hspace{5mm}
1)  $\sqrt{90}=\sqrt{9 \cdot 10}=\sqrt{9} \cdot \sqrt{10}=3\cdot \sqrt{10}$

\vspace{2mm}
\hspace{5mm}
2) $4\cdot \sqrt{7}=\sqrt{4^{2} \cdot 7}=\sqrt{16 \cdot 7}= \sqrt{112}$


\end{flushleft}
}}}
\end{center}
\vspace{0.5cm}

\textbf{Märkmed}\\
\vspace{2mm}
\begin{mdframed}[style=graphpaper]
\vspace{6cm}
\end{mdframed}