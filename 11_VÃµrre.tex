\begin{center}
\fbox{\fbox{\parbox{6.5in}{\centering


\begin{flushleft}
\hspace{5mm}
\textbf{\underline{Võrre}}\\
\hspace{5mm} Võrreks nimetatakse järgmist võrdust:\\
\begin{equation}
\boxed{\dfrac{a}{b}=\dfrac{x}{y}}
\end{equation}

 
\hspace{5mm} Võrre kirjeldab meil tõest võrdust KAHE SUHTE vahel.
\vspace{5mm}

\hspace{5mm} \textbf{Näiteks:} Oletame, et meil on teada, et 3 õuna maksab 5 eurot.\\
\hspace{5mm} Paneme selle esialgu lihtsa võrduse kujul kirja: 
\[ 3 \text{ õuna} \longrightarrow 5 \text{ eur} \]

\hspace{5mm} Nüüd oletame, et tahame teada saada, kui palju maksab näiteks 15 õuna. Kuna 15 õuna hind on meil\\ \hspace{5mm} teadmata, siis võime õunte hinda tähistada tundmatuga $"x"$.\\
\[ 15\text{ õuna} \longrightarrow x \text{ eur} \]\\

\hspace{5mm} Näeme, et kui jagame 15 õuna 3 õunaga, siis on nende kahe arvu vaheliseks suhteks 5. Ehk teisisõnu,\\ \hspace{5mm} näeme, et õunte kogus suurenes 5 korda, $\dfrac{15 \text{ õuna}}{3 \text{ õuna}}= 5$. Kuna õunte kogus suurenes viis korda, siis peaks\\ \hspace{5mm} ka vastavalt suurenema viis korda õunte hind. Ehk $ \dfrac{x \text{ eur}}{5 \text{ eur}}=5$. Et kuigi meil on 15 õuna hind teadmata \\ \hspace{5mm} (x eur), siis vähemalt oleme me teadlikud sellest, et 15 õuna hinna jagamisel 3 õuna hinnaga (milleks\\ \hspace{5mm} oli 5 eur), peaks tulemuseks tulema samuti 5 (st. hindade suhe peaks tulema 5). Ning tänu sellele\\ \hspace{5mm} teadmisele, ei ole 15 õuna hinna leidmine keeruline. Saame järgmise võrrandi: \\

\[ \dfrac{15 \text{ õuna}}{3 \text{ õuna}}=\dfrac{x \text{ eur}}{5 \text{ eur}}   \] \\

\hspace{5mm} Kuna meie eesmärk on leida "x", ehk 15 õuna hind, siis me võiksime sellest "5 eur"st kuidagi lahti\\ \hspace{5mm} saada. Üks võimalus sellest lahti saamiseks, oleks nii, kui me korrutaksime võrduse parema poole 5-ga,\\ \hspace{5mm} kuna siis taanduksid viied ära ja alles jääkski vaid x. Kuid niisama me paremat poolt korrutada ei saa.\\ \hspace{5mm} Tasakaalu säilitamiseks, tuleb meil samamoodi käituda ka võrduse vasaku poolega. Ehk:\\

\begin{equation}
\left.\dfrac{15 }{3 }=\dfrac{x }{5 } \hspace{2mm} \right| \cdot 5
\label{eq11_2}
\end{equation}

\begin{equation}
\dfrac{15 \cdot 5}{3 }=\dfrac{x  \cdot 5}{5 }
\label{eq11_3}
\end{equation}

\begin{equation}
 \dfrac{15 \cdot 5 }{3 }=\dfrac{x \cdot \cancel{5}}{\cancel{5} } \longrightarrow \dfrac{75}{3} = x \longrightarrow 25 = x
 \label{eq11_4}
\end{equation}

\hspace{5mm} Saime, et 15 õuna hind on 25 eur.

\vspace{5mm}

\hspace{5mm} \textbf{\underline{Ristkorrutis}}

\hspace{5mm} Eelnevat lahenduskäiku võib lühidalt kokku võtta mõistega "ristkorrutis".\\
\hspace{5mm} Ehk täpsemalt: kui võrrelda esimest avaldist \ref{eq11_2} ja lõpp tulemust \ref{eq11_4}, siis on märgata, et \\ \hspace{5mm} põhimõtteliselt meil liigub võrduse paremal pool olev 5 lihtsalt vasakule poole risti ülesse.

\end{flushleft}
}}}
\end{center}

\newpage


\begin{center}
\fbox{\fbox{\parbox{6.5in}{\centering


\begin{flushleft}
\hspace{5mm} \textbf{\underline{Alternatiivne viis võrdelise seose ülesannete lahendamiseks}}

\vspace{2mm}

\hspace{5mm} Veidi loomulikum oleks eelmist ülesannet lahendada hoopis järgmiselt:\\

\hspace{5mm} Kuna me teame, et 3 õuna hind on 5 eurot, siis tegelikult on meil võimalik välja arvutada ka ühe õuna\\ \hspace{5mm} hind. Kui ühe õuna hind on teada, siis ei ole raske leida 15 õuna hinda, kuna piisaks vaid sellest, et\\ \hspace{5mm} korrutame ühe õuna hinna soovitud õunte kogusega läbi, ehk praegusel juhul arvuga 15.\\

\hspace{5mm} Järelikult:

\hspace{5mm}
\begin{equation}
3 \text{ õuna} \longrightarrow 5 \text{ eur}
\end{equation}

\begin{equation}
3 \text{ õuna} = 5 \text{ eur} \bigg| : 3
\end{equation}

\begin{equation}
\dfrac{3 \text{ õuna}}{3} = \dfrac{ 5 \text{ eur}}{3}
\end{equation}


\begin{equation}
\dfrac{\cancel{3} \text{ õuna}}{\cancel{3}} = \dfrac{ 5 \text{ eur}}{3}
\end{equation}

\begin{equation}
\dfrac{1 }{1}\text{ õun} = \dfrac{ 5 }{3}\text{ eur} \hspace{2mm} \longrightarrow \hspace{2mm} 1\text{ õun} \approx 1.67\text{ eur}
\end{equation}

\hspace{5mm} Ning viimaks, et leida 15 õuna hind, korrutame mõlemad võrduse pooled 15ga läbi.

\begin{equation}
1\text{ õun} = \dfrac{ 5 }{3}\text{ eur} \bigg| \cdot 15 
\end{equation}

\begin{equation}
1 \text{ õun}\cdot 15 = \dfrac{ 5 }{3}\text{ eur} \cdot 15 \hspace{2mm} \longrightarrow \hspace{2mm} 15 \text{ õun} = \dfrac{ 5 \cdot 15 }{3}\text{ eur}
\end{equation}


\begin{equation}
15 \text{ õuna} = 25 \text{ eur}
\end{equation}

\end{flushleft}
}}}
\end{center}

\vspace{1cm}

\textbf{Märkmed}\\
\vspace{2mm}
\begin{mdframed}[style=graphpaper]
\vspace{7cm}
\end{mdframed}