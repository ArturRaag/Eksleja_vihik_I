\begin{center}
\fbox{\fbox{\parbox{6.5in}{\centering
\begin{flushleft}

\vspace{2mm}
\hspace{5mm}
\textbf{\underline{Mittetäieliku ruutvõrrandi lahendamine}}

\vspace{2mm}
\hspace{5mm}
Ruutvõrrandit, mille lineaarliige $bx$ või vabaliige $c$ on null, nimetatakse \textbf{mittetäielikuks\\ \hspace{5mm} ruutvõrrandiks}.

\vspace{2mm}
\hspace{5mm}
\textbf{Näited:}

\vspace{2mm}
\hspace{5mm}
1) $3x^{2}+5x=0$

\vspace{2mm}
\hspace{5mm}
2) $-7x^{2}+2=0$

\vspace{2mm}
\hspace{5mm}
3) $15x^{2}=0$

\vspace{2mm}
\hspace{5mm}
Kõik kolm võrrandit on mittetäielikud, kuna puudub kas lineaarliige $bx$ või vabaliige $c$ või mõlemad\\ \hspace{5mm} korraga.

\vspace{5mm}
\hspace{5mm}
Nagu peatükkis \ref{37_peatükk} mainiti, igat ruutvõrrandit, mille diskriminant $D$ ei ole nullist väiksem (ehk ei ole\\ \hspace{5mm} negatiivne), saab lahendada valemiga \ref{37_eq5}. Ka mittetäielike ruttvõrrandeid saab sellega lahendada.\\ \hspace{5mm} Kuid selles peatükkis uurime teatuid kavalusi, mis võimaldaksid meil lahendeid leida valemit \ref{37_eq5}\\ \hspace{5mm} kasutamata ning seda ka veidi kiiremini.


\vspace{2mm}
\hspace{5mm}
\textbf{Näited:}

\vspace{2mm}
\hspace{5mm}
\textbf{1)} $9x^{2}-2x=0$

\vspace{2mm}
\hspace{5mm}
Näeme, et mõlemal liikmel on ühiseks teguriks $x$. Toome selle $x$ - i sulgude ette.

\[9x^{2}-2x=0 \hspace{3mm} \longrightarrow \hspace{3mm} x(9x-2)=0  \]

\hspace{5mm}
Kuna võrduse parempool on $0$, siis peab võrduse vasakpool samuti $0$ olema. Vasakule poole\\ \hspace{5mm} saavutaksime nulli, kui \textbf{üks teguritest oleks null}. Ehk, kui sulgude ees olev $x$ oleks 0 (x=0) või\\ \hspace{5mm} sulgude sees olev avaldis $9x-2$ oleks null $(9x-2=0)$.

\vspace{2mm}
\hspace{5mm}
Järelikult meie lahendid on:

\[ \begin{tabular}{c}
$x_{1}=0$\\
$9x-2=0 \hspace{3mm} \longrightarrow \hspace{3mm} 9x=2 \hspace{3mm} \longrightarrow \hspace{3mm} x_{2}=\dfrac{2}{9}$
\end{tabular} \]

\vspace{2mm}
\hspace{5mm}
Nii $x_{1}=0$ kui ka $x_{2}=\dfrac{2}{9}$ korral saame võrrandi vasakupoole võrduma paremapoolega.

\vspace{2mm}
\hspace{5mm}
Kontroll: \[ \begin{tabular}{l}
VP\textsubscript{1} $= 0(9 \cdot 0 -2)= 0(0-2)=0(-2)=0$\\
PP\textsubscript{1} $= 0$\\
VP\textsubscript{1}$=$PP\textsubscript{1} \vspace{5mm}\\

VP\textsubscript{2}$=\dfrac{2}{9} \left( 9\cdot \dfrac{2}{9} - 2 \right) = \dfrac{2}{9} \left( \dfrac{18}{9}-2 \right) = \dfrac{2}{9} \left( 2-2 \right)=\dfrac{2}{9}(0)=0$\\
PP\textsubscript{2}$=0$\\
VP\textsubscript{2}$=$PP\textsubscript{2}
\end{tabular} \]

\vspace{2mm}
\hspace{5mm}
Vastus: Ruutvõrrandi $9x^{2}-2x=0$ lahendid on $x_{1}=0$ ja $x_{2}=\dfrac{2}{9}$
\end{flushleft}
}}}
\end{center}

\pagebreak
\begin{center}
\fbox{\fbox{\parbox{6.5in}{\centering
\begin{flushleft}

\vspace{2mm}
\hspace{5mm}
\textbf{2)} $2x^{2}-18=0$

\vspace{2mm}
\hspace{5mm}
Siin enam $x$ - i sulgude ette tuua ei saa. Kuid saame viia kõik arvud paremalepoole, viia ruutliige\\ \hspace{5mm} taandatud kujule (st, et $x^{2}$ ees oleks arv $1$) ning panna mõlemad võrduse pooled ruutjuure alla\\ \hspace{5mm} (kuna $\sqrt{x^{2}}=x$, mida otsimegi). Ehk:

\[ 2x^{2}-18=0 \hspace{3mm} \longrightarrow \hspace{3mm} 2x^{2}=18 \hspace{3mm} \longrightarrow \hspace{3mm} x^{2}=\dfrac{18}{2} \hspace{3mm} \longrightarrow \hspace{3mm} x^{2}=9\]

\hspace{5mm}
Paneme mõlemad võrduse pooled ruutjuure alla.

\hspace{5mm}
NB! Kuna $x$ on ruudus, siis tulemus, mis me saame tohib olla nii positiivne kui ka negatiivne, sest:\\ \hspace{5mm} $(-3)^{2}=9$ kui ka $(3)^{2}=9$.

\[\sqrt{x^{2}}=\sqrt{9} \hspace{3mm} \longrightarrow \hspace{3mm} \pm x=\sqrt{9}=3 \]

\hspace{5mm}
Järelikult meie lahendid on:
\[ \begin{tabular}{l}
$x_{1}=3$\\
$x_{2}=-3$
\end{tabular} \]

\vspace{2mm}
\hspace{5mm}
\textbf{3)} $2x^{2}+18=0$

\vspace{2mm}
\hspace{5mm}
Tehe on sarnane eelmisega, kuid siin näites on meil kahe liikme vahel miinuse asemel pluss. Siin selgub,\\ \hspace{5mm} et juure alla saame negatiivse arvu, millest ruutjuurt võtta ei saa. Sellisel juhul lahendid aga puuduvad.

\[2x^{2}+18=0 \hspace{3mm} \longrightarrow \hspace{3mm} 2x^{2}=-18
 \hspace{3mm} \longrightarrow \hspace{3mm} x^{2}=-\dfrac{18}{2} \hspace{3mm} \longrightarrow \hspace{3mm} x^{2}=-9 \hspace{3mm} \longrightarrow \hspace{3mm} \sqrt{x^{2}}=\sqrt{-9} \]

\[ \pm x = \sqrt{-9} \hspace{3mm} \longrightarrow \hspace{3mm} \left[ \begin{tabular}{c}
Negatiivsest arvust\\
ruutjuurt võtta ei saa.\\
Lahendid puuduvad.
\end{tabular} \right] \] 

\vspace{2mm}
\hspace{5mm}
\textbf{4)} $5x^{2}=0$

\vspace{2mm}
\hspace{5mm}
Siin puuduvad nii lineaarliige kui ka vabaliige. On ilmselgelt näha, et ainuke lahend oleks hetkel\\ \hspace{5mm} $x_{1}=x_{2}=0$ kuna see on ainuke arv, mille puhul oleks meil mõlemad võrduse pooled nullid. Aga võime\\ \hspace{5mm} harjutamise eesmärgil teha seda ka pikkemalt. Viime ruutliikme taandatud kujule ja paneme mõlemad\\ \hspace{5mm} võrduse pooled ruutjuure alla.

\[ 5x^{2}=0 \hspace{2mm} \bigg| :5 \]

\[x^{2}=\dfrac{0}{5} \hspace{3mm} \longrightarrow \hspace{3mm} x^{2}=0  \]

\[\sqrt{x^{2}}=\sqrt{0} \hspace{3mm} \longrightarrow \hspace{3mm} \pm x=0 \]

\hspace{5mm}
Järelikult lahendid ongi $x_{1}=x_{2}=0$.

\end{flushleft}
}}}
\end{center}
\vspace{0.5cm}

\textbf{Märkmed}\\
\vspace{2mm}
\begin{mdframed}[style=graphpaper]
\vspace{2cm}
\end{mdframed}