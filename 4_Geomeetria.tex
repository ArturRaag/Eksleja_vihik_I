\begin{center}
\fbox{\fbox{\parbox{6.5in}{\centering
\begin{flushleft}


\vspace{5mm}

\hspace{5mm} \textbf{\underline{Ruut:}}
\vspace{5mm}

\hspace{5mm} Ümbermõõt: \fbox{$P=4\cdot a$} \hspace{5mm} Pindala: \fbox{$S=a^2 = a\cdot a = a\cdot h$}\\
\vspace{2mm}
\hspace{5mm} kus P - ümbermõõt, S - pindala, a - ruudu külg, h - ruudu kõrgus (NB! a = h).
\vspace{5mm}

\hspace{5mm} \textbf{\underline{Ristkülik:}}
\vspace{5mm}

\hspace{5mm} Ümbermõõt: \fbox{$P=2\cdot(a+h)=a+h+a+h$} \hspace{5mm} Pindala: \fbox{$S=a\cdot h$}\\
\vspace{2mm}
\hspace{5mm} kus P - ümbermõõt, S - pindala, a - ristküliku alus ning h - ristküliku kõrgus. Ristküliku puhul $a\neq h$.
\vspace{5mm}

\hspace{5mm} \textbf{\underline{Kolmnurk:}}
\vspace{5mm}

\hspace{5mm} Ümbermõõt: \fbox{$P=b+c+d$} \hspace{5mm} Pindala: \fbox{$S=\dfrac{a\cdot h}{2}$}\\
\vspace{2mm}
\hspace{5mm} kus $P$ - ümbermõõt, $b$, $c$ ja $d$ - kolmnurga erinevad küljed, $S$ - pindala, $a$ - kolmnurga alus, \\
\hspace{5mm} $h$ - kolmnurga kõrgus.\\
\vspace{5mm}
\hspace{5mm} Kolmnurga liigitamine [ Külgede järgi ]:\\
\vspace{2mm}
\hspace{5mm} 1) Erikülgne - kõik küljed on erineva pikkusega.\\
\vspace{2mm}
\hspace{5mm} 2) Võrdkülgne - kõik küljed on võrdse pikkusega.\\
\vspace{2mm}
\hspace{5mm} 3) Võrdhaarne - vaid kaks külge on võrdsed. Alus on teise pikkusega.\\
\vspace{5mm}
\hspace{5mm} Kolmnurga liigitamine [ Nurkade järgi ]:\\
\vspace{2mm} 
\hspace{5mm} 1) Teravnurkne - Kõik nurgad on alla 90-kraadi. Ehk: $\alpha ,\beta$ ja $\gamma$ nurgad $< 90\degree$.\\
\vspace{2mm}
\hspace{5mm} 2) Nürinurkne - VÄHEMALT ÜKS nurkadest ületab 90-kraadi. Ehk: $\alpha$ või $\beta$ või $\gamma > 90\degree$.\\
\vspace{2mm}
\hspace{5mm} 3) Täisnurkne - VÄHEMALT ÜKS nurkadest on TÄPSELT 90-kraadi. Ehk: $\alpha$ või $\beta$ või $\gamma = 90\degree$.\\
\vspace{5mm}

\hspace{5mm} \textbf{\underline{Ring:}}
\vspace{5mm}

\hspace{5mm} Diameeter ehk läbimõõt ning raadius: \fbox{$d=2\cdot r$} \hspace{5mm} \fbox{$r=\dfrac{d}{2}$}\\
\vspace{2mm}

\hspace{5mm} Ümbermõõt ehk ringjoon: \fbox{$C=2\cdot \pi \cdot r=d\cdot \pi$} \hspace{5mm} Pindala: \fbox{$S=\pi \cdot r^2=\pi \cdot r \cdot r$}\\
\vspace{2mm}
\hspace{5mm} kus C - ringjoon, $\pi \approx 3.14$, r - ringi raadius(pool diameetrist), d - ringi diameeter (ehk läbimõõt),\\
\hspace{5mm} S - pindala.
\end{flushleft}
}}}
\end{center}

\newpage



\begin{center}
\fbox{\fbox{\parbox{6.5in}{\centering
\begin{flushleft}

\vspace{2mm}
\hspace{5mm}
\textbf{\underline{Trapets}}

\vspace{2mm}
\hspace{5mm}
Üldine trapetsi pindala valem: $\boxed{S=\dfrac{a+b}{2}\cdot h}$

\vspace{2mm}
\hspace{5mm}
kus $a$ - üks trapetsi alustest, $b$ - trapetsi teine alus, $h$ - trapetsi kõrgus.

\vspace{5mm}
\hspace{5mm}
Kesklõigu kaudu avaldatud trapetsi pindala valem: $\boxed{S=k\cdot h}$

\vspace{2mm}
\hspace{5mm}
kus $k$ - trapetsi kesklõigu pikkus, $h$ - trapetsi kõrgus.


\end{flushleft}
}}}
\end{center}

\textbf{Märkmed}\\
\vspace{2mm}
\begin{mdframed}[style=graphpaper]
\vspace{15cm}
\end{mdframed}