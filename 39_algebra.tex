\begin{center}
\fbox{\fbox{\parbox{6.5in}{\centering
\begin{flushleft}

\vspace{2mm}
\hspace{5mm}
\textbf{\underline{Põhivalemid}}

\vspace{2mm}
\hspace{5mm}
\textbf{Varasemalt õpitud valemid:}

\begin{equation}
\label{39_eq1}
\boxed{(a+b)^{2}=(-a-b)^{2}=a^{2}+2ab+b^{2}}
\end{equation}


\begin{equation}
\label{39_eq2}
\boxed{(a-b)^{2}=(b-a)^{2}=a^{2}-2ab+b^{2}}
\end{equation}


\begin{equation}
\label{39_eq3}
\boxed{a^{2}-b^{2}=(a-b)(a+b)}
\end{equation}


\vspace{5mm}
\hspace{5mm}
\textbf{Ruutkolmliikme tegurdamine:}

\begin{equation}
\label{39_eq4}
\boxed{ax^{2}+bx+c=a(x-x_{1})(x-x_{2})}
\end{equation}


\vspace{2mm}
\hspace{5mm}
kus $x_{1}$ ja $x_{2}$ on ruutvõrrandi lahendid.

\vspace{5mm}
\hspace{5mm}
\textbf{Sarnaste liikmete liitmine ja korrutamine:}

\begin{equation}
\label{39_eq5}
\boxed{a+a=2 \cdot a=2a}
\end{equation}

\begin{equation}
\label{39_eq6}
\boxed{a \cdot a = a^{2}}
\end{equation}

\hspace{5mm}
Samuti tulevad kasuks valemid peatükkidest \ref{2_peatükk}, \ref{3_peatükk}, \ref{6_peatükk} ja \ref{35_peatükk} (kahjuks kõik siia ära ei mahu).

\vspace{5mm}
\hspace{5mm}
\textbf{Näited:}

\vspace{2mm}
\hspace{5mm}
\textbf{1)} Tegurda avaldis $10x^{2}+3x-4$

\vspace{2mm}
\hspace{5mm}
Lahendid saab leida valemiga \ref{37_eq5}.

\[ x_{1,2}= \dfrac{-3 \pm \sqrt{3^{2}-4 \cdot 10 \cdot (-4)}}{2 \cdot 10} = \dfrac{-3 \pm 13}{20}= \begin{tabular}{l}
$x_{1}=\dfrac{-3-13}{20}=-\dfrac{16}{20}=-\dfrac{4}{5}$\\
\\
$x_{2}=\dfrac{-3+13}{20}=\dfrac{10}{20}=\dfrac{1}{2}$
\end{tabular} \]

\hspace{5mm}
Kasutades valemit \ref{39_eq4} saame nüüd ruutvõrrandit tegurdada.

\[ 10x^{2}+3x-4 = 10\Big( x-\left(-\dfrac{4}{5} \right) \Big)\Big( x-\dfrac{1}{2} \Big)= 10 \Big( x + \dfrac{4}{5} \Big) \Big( x-\dfrac{1}{2} \Big)= 5 \Big( x+ \dfrac{4}{5} \Big) \cdot 2 \Big( x-\dfrac{1}{2} \Big)= \]

\[=(5x+4)(2x-2) \]

\vspace{2mm}
\hspace{5mm}
\textbf{2)} Lihtsusta järgmine murd: $\dfrac{x^{2}-9}{x^{2}-6x+9}$ 


\vspace{2mm}
\hspace{5mm}
\[ \dfrac{x^{2}-9}{x^{2}-6x+9} = \left[ \begin{tabular}{c}
Tegurdame lugeja valemiga \ref{39_eq3}\\
ja nimetaja valemiga \ref{39_eq2}
\end{tabular} \right] = \dfrac{(x-3)(x+3)}{(x-3)(x-3)}= \dfrac{\cancel{(x-3)}(x+3)}{\cancel{(x-3)}(x-3)}=\dfrac{(x+3)}{(x-3)} \]

\end{flushleft}
}}}
\end{center}

\pagebreak
\begin{center}
\fbox{\fbox{\parbox{6.5in}{\centering
\begin{flushleft}

\vspace{2mm}
\hspace{5mm}
\textbf{3)} Lihtsusta avaldis $\overbrace{\dfrac{x^{2}+y^{2}}{x^{2}-y^{2}}\cdot \dfrac{x-y}{x+y}}^{a} \overbrace{: \dfrac{5x^{2}+5y^{2}}{x^{2}+2xy+y^{2}}}^{b}$

\vspace{5mm}
\hspace{5mm}
a) $\dfrac{x^{2}+y^{2}}{x^{2}-y^{2}}\cdot \dfrac{x-y}{x+y}= \dfrac{(x^{2}+y^{2}) \cdot (x-y)}{(x^{2}-y^{2})\cdot (x+y)}=\left[ \begin{tabular}{c}
Tegurdame $(x^{2}-y^{2})$\\
kasutades valemit\\
\ref{39_eq3}
\end{tabular} \right]=\dfrac{(x^{2}+y^{2}) (x-y)}{(x+y)(x-y)(x+y)}= $

\vspace{5mm}
\hspace{5mm}
$=\dfrac{(x^{2}+y^{2}) \cancel{(x-y)}}{(x+y)\cancel{(x-y)}(x+y)} = \dfrac{x^{2}+y^{2}}{(x+y)^{2}}$


\vspace{5mm}
\hspace{5mm}
b) $\dfrac{x^{2}+y^{2}}{(x+y)^{2}}:\dfrac{5x^{2}+5y^{2}}{x^{2}+2xy+y^{2}} = \left[ \begin{tabular}{c}
Tehte teises murrus saab lugejat\\
tegurdada tuues ühise kordaja \\
sulgude ette: $(kx+ky)=k(x+y)$\\
Nimetaja saab tegurdada valemiga \ref{39_eq1}.
\end{tabular} \right]=$

\vspace{5mm}
\hspace{5mm}
$=\dfrac{x^{2}+y^{2}}{(x+y)^{2}}:\dfrac{5(x^{2}+y^{2})}{(x+y)^{2}}=\left[\begin{tabular}{c}
Murdude\\
jagamine.
\end{tabular} \right]=\dfrac{x^{2}+y^{2}}{(x+y)^{2}} \ \cdot \dfrac{(x+y)^{2}}{5(x^{2} + y^{2})} = \dfrac{\cancel{(x^{2}+y^{2})} \cancel{(x+y)^{2}}}{5(\cancel{x^{2} + y^{2}}) \cancel{(x+y)^{2}}} = \dfrac{1}{5}$



\vspace{5mm}
\hspace{5mm}
\textbf{4)}  Lihtsusta avaldis: $\overbrace{\left( \dfrac{2}{x} - \dfrac{1}{x+1} \right)}^{a} \overbrace{\cdot \dfrac{x^{2}+x}{x^{2}+x-2}}^{b}$


\vspace{2mm}
\hspace{5mm}
a) $ \dfrac{2}{x} - \dfrac{1}{x+1} = \left[ \begin{tabular}{c}
Ristlaiendus.\\
Peatükk \ref{3_peatükk}.
\end{tabular}
\right] = \dfrac{2}{x}^{(x+1} - \dfrac{1}{x+1}^{(x} = \dfrac{2(x+1)}{x(x+1)} - \dfrac{1\cdot x}{x(x+1)}=\dfrac{2(x+1)-x}{x(x+1)} = $

\vspace{5mm}
\hspace{5mm}
$=\dfrac{2x+2-x}{x(x+1)}=\dfrac{x+2}{x(x+1)}$

\vspace{5mm}
\hspace{5mm}
b) $\dfrac{x+2}{x(x+1)}\cdot \dfrac{x^{2}+x}{x^{2}+x-2}= \left[ \begin{tabular}{c}
Tegurdame teise murru lugeja\\
tõstes ühise kordaja sulgude ette.\\
Teise murru nimetaja saame tegurdada\\
kas valemiga \ref{39_eq4} või Viete'i\\
teoreemiga (\ref{viete})
\end{tabular} \right]=$

\vspace{5mm}
\hspace{5mm}
$= \dfrac{x+2}{x(x+1)}=\dfrac{(x+2) \cdot x(x+1)}{x(x+1)\cdot (x-1)(x+2)}= \dfrac{(\cancel{x+2}) \cdot \cancel{x(x+1)}}{ \cancel{x(x+1)}\cdot (x-1)(\cancel{x+2})}=\dfrac{1}{x-1}$
\end{flushleft}
}}}
\end{center}

\vspace{0.5cm}

\textbf{Märkmed}\\
\vspace{2mm}
\begin{mdframed}[style=graphpaper]
\vspace{3cm}
\end{mdframed}