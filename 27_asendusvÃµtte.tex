\begin{center}
\fbox{\fbox{\parbox{6.5in}{\centering
\begin{flushleft}

\vspace{2mm}
\hspace{5mm}
\textbf{\underline{Võrrandisüsteemi lahendamine asendusvõttega}}

\vspace{2mm}
\hspace{5mm}
Proovime see kord lahendada sama võrrandisüsteemi (\ref{25_eq3}), kuid seekord asendusvõttega.

\vspace{2mm}
\hspace{5mm}
Asendusvõtte eesmärk on avaldada ühest võrrandist kumbki tundmatutest (kujul $x=...$ või $y=...$),\\ \hspace{5mm} asendada see teisse võrrandisse ning leida ühe tundmatuga võrrandi lahend.

\[ \begin{cases}
2x+5y=10\\
-3x+15y=1
\end{cases} \]

\hspace{5mm}
Õnneks oleme me juba võrrandid avaldanud peatükkis \ref{25_peatükk} avaldis \ref{25_eq4}. Vajadusel saab mainitud\\ \hspace{5mm} peatükkist avaldamise käiku järgi vaadata. Süsteem nägi lõpuks välja nõnda: 

\[ \begin{cases}
y= 2- \dfrac{2x}{5}\\
y=\dfrac{1}{15}+\dfrac{x}{5}
\end{cases} \] 

\hspace{5mm}
Asendame esimese võrrandi (esimese $y$ - i) teise võrrandi sisse (teise $y$ - i sisse). Saame järgmise võrduse:

\[ 2-\dfrac{2x}{5}=\dfrac{1}{15}+\dfrac{x}{5} \]

\hspace{5mm}
Saime $y$ - ist lahti ning nüüd saame leida $x$ - i.

\hspace{5mm}
Viime $x$-idega liikmed vasakule poole ja muud liikmed paremale poole võrdust.

\[ -\dfrac{2x}{5}-\dfrac{x}{5}=\dfrac{1}{15}-2 \]

\hspace{5mm}
Avaldame $x$-i.

\[-\dfrac{2x}{5}-\dfrac{x}{5}=\dfrac{1}{15}-2 \hspace{3mm} \longrightarrow \hspace{3mm} \dfrac{-2x-x}{5}=\dfrac{1}{15}^{(1}-\dfrac{2}{1}^{(15} \hspace{3mm} \longrightarrow \hspace{3mm} \dfrac{-3x}{5}=\dfrac{1-30}{15} \hspace{3mm} \longrightarrow ... \]


\[... \longrightarrow \dfrac{-3x}{5}=-\dfrac{29}{15} \]

\hspace{5mm}
Et nüüd vasakule poole jääks ainult $x$, tuleb meil kogu võrrand läbi korrutada $5$ga (et\\ \hspace{5mm} nimetajas olev $5$ ära taanduks), ning jagada läbi $-3$ga (siis taandub ära ka lugejas olev $3$ ning märk\\ \hspace{5mm} muutub plussiks). 

\vspace{2mm}
\hspace{5mm}
Seda saab teha ühe kaupa, kuid kiirem oleks seda korraga teha korrutades murruga $-\dfrac{5}{3}$:

\[ \dfrac{-3x}{5}=-\dfrac{29}{15} \hspace{2mm} \bigg| \cdot \left( -\dfrac{5}{3} \right) \]

\[ \dfrac{-3x}{5}\cdot \left(- \dfrac{5}{3} \right)= -\dfrac{29}{15} \cdot \left( - \dfrac{5}{3} \right) \hspace{3mm} \longrightarrow \hspace{3mm} \dfrac{\cancel{3} \cdot \cancel{5} \cdot x}{\cancel{5} \cdot \cancel{3}}=\dfrac{29 \cdot 5}{15 \cdot 3} \]

\hspace{5mm}
Saame: 
\[ x= \dfrac{29}{9} \approx 3.2 \]

\end{flushleft}
}}}
\end{center}

\pagebreak

\begin{center}
\fbox{\fbox{\parbox{6.5in}{\centering
\begin{flushleft}

\vspace{2mm}
\hspace{5mm}
Nüüda asendame leitud tundmatu $x=\dfrac{29}{9}$ kummagisse esialgsesse võrrandisse ja leiame $y$-i. Võtame\\ \hspace{5mm} näiteks võrrandi $2x+5y=10$. Asendame $x=\dfrac{29}{9}$:

\[ 2 \cdot \dfrac{29}{9}+5y=10 \hspace{3mm} \longrightarrow \hspace{3mm} 5y=10-\dfrac{2 \cdot 29}{9}\]

\[ 5y=10-\dfrac{2 \cdot 29}{9} \hspace{2mm} \bigg| : (5) \]

\[ \dfrac{5y}{5} = \dfrac{10}{5} - \dfrac{2 \cdot 29}{9 \cdot 5} \]

\[ \dfrac{\cancel{5} \cdot y}{\cancel{5}} = \dfrac{\cancel{10}}{\cancel{5}} - \dfrac{2 \cdot 29}{9 \cdot 5} \]

\hspace{5mm}
Alles jäi: 

\[ y= 2-\dfrac{58}{45} \]

\hspace{5mm}
Ehk:
\[ y=\dfrac{32}{45} \approx
0.71 \]

\hspace{5mm}
Saime samad lahendid, mis liitmisvõttegagi:

\[ \begin{cases}
x=\dfrac{29}{9}\\
\\
y=\dfrac{32}{45}
\end{cases} \]
\end{flushleft}
}}}
\end{center}

\vspace{0.5cm}

\textbf{Märkmed}\\
\vspace{2mm}
\begin{mdframed}[style=graphpaper]
\vspace{5cm}
\end{mdframed}