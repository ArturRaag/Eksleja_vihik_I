\begin{center}
\fbox{\fbox{\parbox{6.5in}{\centering
\begin{flushleft}

\vspace{2mm}
\hspace{5mm}
NB! Enne seda peatükki uurimist on soovituslik selgeks saada peatükkid \ref{20} ja \ref{21}.

\vspace{2mm}
\hspace{5mm}
\textbf{\underline{Hulkliige}}

\vspace{2mm}
\hspace{5mm}
Hulkliikmeks nimetatakse avaldist, mis koonseb üksliikmete summast.

\vspace{2mm}
\hspace{5mm} 
Ehk kui meil on näiteks esimene üksliige $6xy^{3}$ ja teine üksliige on $-7ab^{4}c$, siis nende liitmisel saame\\ \hspace{5mm} hulkliikme, mis näeb välja nõnda: $6xy^{3}-7ab^{4}c$.

\vspace{2mm}
\hspace{5mm}
Veel mõned \textbf{näited} hulkliikmetest:

\vspace{2mm}
\hspace{5mm}
1) $2a-9$

\vspace{2mm}
\hspace{5mm}
2) $3x^{2}z+12i$

\vspace{2mm}
\hspace{5mm}
3) $-14f^{3}+11ab-2x$

\vspace{2mm}
\hspace{5mm}
Pange tähele, et hulkliikmeid koondada ei saa, kuna nende üksliikmed ei ole sarnased!

\hspace{5mm}
Ehk \textbf{näiteks} avaldis $3x-9x$ \textbf{ei ole} hulkliige, kuna seda saab koondada: $3x-9x=(3-9)x=-6x$\\ \hspace{5mm} ning $-6x$ on üksliige.

\vspace{2mm}
\hspace{5mm}
\textbf{\underline{Hulkliikmete liitmine ja lahutamine}}

\vspace{2mm}
\hspace{5mm}
Et vältida võimaliku märgivea tekkimist, on soovituslik kirjutada hulkliikmed alati sulgudes. Kui\\ \hspace{5mm} hulkliikmeid liidetakse, siis tegelikult võib sulud ära jätta, kuna "+" märk sulu ees, meil märke sulu\\ \hspace{5mm} sees ei muudaks. Küll aga kui me hulkliikmeid lahutame, siis sulge avades tuleb arvestada, et \textbf{miinus\\ \hspace{5mm} märk sulu ees muudab märgid sulu sees}. Lisaks, kui leidub sarnaseid liikmeid, siis ära unusta\\ \hspace{5mm} koondada!

\vspace{2mm}
\hspace{5mm}
\textbf{Näited:}

\vspace{2mm}
\hspace{5mm}
1) $(2x+5)+(6x-7)=\left[ \begin{tabular}{c}
Sulgude ees on +\\
Sulge avades jäävad\\
märgid samaks
\end{tabular} \right]= 2x+5+6x-7 = \left[ \begin{tabular}{c}
Liidame sarnased\\
liikmed kokku.
\end{tabular} \right] = 8x-2$

\vspace{2mm}
\hspace{5mm}
2) $(3xy^{8}-5k)-(8xy^{8}-9k+4z)= \left[ \begin{tabular}{c}
Esimese sulu ees $+$\\
Teise sulu ees $-$\\
Miinus muudab \\
märgid sulu sees!
\end{tabular} \right]= 3xy^{8}-5k - 8xy^{8}+9k-4z  =$\\
\hspace{10mm}
$=\left[\begin{tabular}{c}
Liidame sarnased\\
liikmed kokku.
\end{tabular} \right]= -5xy^{8}+4k -4z $

\vspace{5mm}
\hspace{5mm}
\textbf{\underline{Hulkliikme korrutamine üksliikmega}}

\vspace{2mm}
\hspace{5mm}
Hulkliikme korrutamisel üksliikmega, tuleb meil iga hulkliikme liige korrutada ükshaaval üksliikmega\\ \hspace{5mm} läbi. Kusjuures märgireegleid tuleb samuti silmas pidada.

\vspace{2mm}
\hspace{5mm}
Üldine valem:

\begin{equation}
\label{22_eq1}
a(b+c)=ab+ac
\end{equation}

\end{flushleft}
}}}
\end{center}

\pagebreak

\begin{center}
\fbox{\fbox{\parbox{6.5in}{\centering
\begin{flushleft}

\vspace{2mm}
\hspace{5mm}
\textbf{Näited:}

\vspace{2mm}
\hspace{5mm}
1) $3(2x-7y)=3\cdot 2x + 3 \cdot (-7y)=6x-21y$

\vspace{2mm}
\hspace{5mm}
2) $7x(2x+3)-5(x^{2}+xz)=7x\cdot 2x + 7x \cdot 3 +(-5)\cdot x^{2}+(-5)\cdot xz=14x^{2}+21x-5x^{2}-5xz=$

\vspace{2mm}
\hspace{10mm}
$=9x^{2}-5xz+21x$


\vspace{2mm}
\hspace{5mm}
3) $-5x^{2}(-8k+10y-z^{3})=(-5x^{2})\cdot (-8k)+(-5x^{2})\cdot 10y +(-5x^{2})\cdot (-z^{3})=40x^{2}k-50x^{2}y+5x^{2}z^{3}$

\vspace{5mm}
\hspace{5mm}
\textbf{\underline{Hulkliikme jagamine üksliikmega}}

\vspace{2mm}
\hspace{5mm}
Hulkliikme jagamisel üksliikmega, jagame kõik hulkliikme liikmed sulgude ees oleva üksliikmega.

\vspace{2mm}
\hspace{5mm}
Seda saab teha kahel erineval viisil:

\begin{equation}
\label{22_eq2}
(a+b-c):d=a:d+b:d-c:d
\end{equation}

\begin{equation}
\label{22_eq3}
(a+b-c):d=\dfrac{a+b-c}{d}=\dfrac{a}{d}+\dfrac{b}{d} -\dfrac{c}{d}
\end{equation}

\vspace{2mm}
\hspace{5mm}
\textbf{Näited:}

\vspace{2mm}
\hspace{5mm}
1) $(4a-8c):4=4a:4-8c:4=1a-2c=a-2c$

\vspace{2mm}
\hspace{5mm}
2) $(25xy+3y^{3}):5y=\dfrac{25xy+3y^{3}}{5y}=\dfrac{25xy}{5y}+\dfrac{3y^{3}}{5y}=\left[ Taandame \right]=\dfrac{\cancel{25}x\cancel{y}}{\cancel{5y}}+\dfrac{3y^{\cancel{3}}}{5\cancel{y}}=\dfrac{5x}{1}+\dfrac{3y^{2}}{5}=$

\vspace{2mm}
\hspace{10mm}
$=5x+\dfrac{3y^{2}}{5}$

\end{flushleft}
}}}
\end{center}

\vspace{0.5cm}

\textbf{Märkmed}\\
\vspace{2mm}
\begin{mdframed}[style=graphpaper]
\vspace{9cm}
\end{mdframed}