\begin{center}
\fbox{\fbox{\parbox{6.5in}{\centering
\begin{flushleft}

\hspace{5mm}
\textbf{\underline{Aritmeetiline keskmine $\bar{x}$}}

\hspace{5mm}
\vspace{2mm}

\begin{equation}
\label{eq16_1}
\boxed{\bar{x} = \dfrac{S}{n}}
\end{equation}

\vspace{2mm}
\hspace{5mm}
kus $S$ on kõikide elementide \textbf{summa}, $n$ on elementide hulk, ehk mitu elementi meil on ning $\bar{x}$ on\\ \hspace{5mm} aritmeetiline keskmine.

\vspace{2mm}
\hspace{5mm}
\textbf{Näide:}

\vspace{2mm}
\hspace{5mm}
Uurime tüüpilist näidet, kuidas keskmist hinnet välja arvutada.

\vspace{2mm}
\hspace{5mm}
Olgu meil õpilane järgmiste hinnetega:

\vspace{2mm}
\hspace{5mm}
\begin{tabular}{|c|c|c|c|c|c|c|c|}
	\hline
	2 & 3 & 3 & 4 & 5 & 5 & 3 & 4\\
	\hline
\end{tabular}

\vspace{2mm}
\hspace{5mm}
Esmalt leiame elementide summa $S$. Selle jaoks liidame lihtsalt kõik hinded kokku.

\begin{equation}
\label{eq16_2}
S = 2+3+3+4+5+5+3+4 = 29
\end{equation}

\hspace{5mm}
Nüüd loendame, mis on meie elementide hulk $n$, ehk mitu hinnet meil on. Loendades saame, et $n$ = 8.

\vspace{2mm}
\hspace{5mm}
Kasutades valemit \ref{eq16_1}, saame, et meie õpilase keskmine hinne on:\\

\begin{equation}
\label{eq16_3}
\bar{x}=\dfrac{29}{8} = 3.625
\end{equation}

\vspace{2mm} 
\hspace{5mm}
Kuna koolis meil kümnendmurrulisi hindeid olemas ei ole, siis ümardatakse need ikkagi lähimaks\\ \hspace{5mm} täisarvuks. Ehk praegusel juhul oleks meil hinne 4.

\vspace{5mm}
\hspace{5mm}
\textbf{\underline{Mediaan $Me$}}

\vspace{2mm}
\hspace{5mm}
Mediaan on järjestatud elementide reas keskmise liikme väärtus.

\vspace{2mm}
\hspace{5mm}
Ehk kui uurida edasi meie hinnete tabelit, siis järjestame kõik hinded kasvavas järjekorras.

\vspace{2mm}
\hspace{5mm}
\begin{tabular}{|c|c|c|c|c|c|c|c|}
	\hline
	2 & 3 & 3 & 3 & 4 & 4 & 5 & 5\\
	\hline
\end{tabular}

\vspace{2mm}
\hspace{5mm}
Nüüd leiame väärtuse, mis on täpselt tabeli keskel. Näeme, et kuna tabelis on meil paarisarv elemente,\\ \hspace{5mm} siis keskmist on raske määrata, kuna tabeli keskel on arvud 3 ja 4. Õnneks saame siin kasutada \\ \hspace{5mm} aritmeetilise keskmise valemit \ref{eq16_1}.

\vspace{2mm}
\hspace{5mm}
\begin{equation}
\label{eq16_4}
\bar{x} = \dfrac{3+4}{2} = 3.5
\end{equation}

\vspace{2mm}
\hspace{5mm}
Järelikult on ka mediaan $Me=3.5$.

\vspace{2mm}
\hspace{5mm}
Teema jätkub järgmisel lehel...


\end{flushleft}
}}}
\end{center}





\begin{center}
\fbox{\fbox{\parbox{6.5in}{\centering
\begin{flushleft}


\vspace{5mm}
\hspace{5mm}
\textbf{\underline{Mood $Mo$}}

\vspace{2mm}
\hspace{5mm}
Mood on kõige sagedamini esinev väärtus.

\vspace{2mm}
\hspace{5mm}
\begin{tabular}{|c|c|c|c|c|c|c|c|}
	\hline
	2 & 3 & 3 & 3 & 4 & 4 & 5 & 5\\
	\hline
\end{tabular}

\vspace{2mm}
\hspace{5mm}
Kui vaatame enda eelnevalt mainitud hindeid, siis on näha, et hinne $2$ esineb meil vaid ühe korra,\\ \hspace{5mm} hinne $3$ esineb kolm korda, hinded $4$ ja $5$ mõlemad kaks korda. 

\vspace{2mm}
\hspace{5mm} Kuna hinne $3$ esines meil kõige sagedamini, siis on mood $Mo=3$.

\vspace{5mm}
\hspace{5mm}
\textbf{\underline{Sagedustabel}}

\vspace{2mm}
\hspace{5mm}
Hetkel oli meil hindeid päris vähe, mille tõttu ei olnud nende uurimine üldsegi tülikas. Kuid võib\\ \hspace{5mm} esineda olukordi, kus hindeid või muid andmeid on väga palju ning nende loendamine, keskmise,\\ \hspace{5mm} moodi ja mediaani leidmine võib muutuda märkimisväärselt raskemaks. Sellisel juhul tuleb appi\\ \hspace{5mm} \textbf{sagedustabel}, mis annab andmetest hea kiire ülevaate.

\vspace{2mm}
\hspace{5mm}
Loome enda hinnetest sagedustabeli. Selle jaoks koostame kaherealise tabeli, kus esimesel real on meil\\ \hspace{5mm} hinnete kategooriad (hinne 1, hinne 2, hinne 3 jne) ning teisel real on nende hinnete esinemissagedused.\\ \hspace{5mm}  Ehk...

\vspace{2mm}
\hspace{5mm}
\begin{tabular}{|c|c|c|c|c|c|c|}
	\hline
	Hinne: & 1 & 2 & 3 & 4 & 5 & Kokku\\
	\hline
	Sagedused: & 0 & 1 & 3 & 2 & 2 & 8\\
	\hline	
\end{tabular}

\vspace{2mm}
\hspace{5mm}
Viimane veerg ütleb meile lihtsalt mitu erinevat hinnet meil kokku oli: (0+1+3+2+2=8)
\end{flushleft}
}}}
\end{center}


\vspace{0.5cm}

\textbf{Märkmed}\\
\vspace{2mm}
\begin{mdframed}[style=graphpaper]
\vspace{9cm}
\end{mdframed}
