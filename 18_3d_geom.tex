\begin{center}
\fbox{\fbox{\parbox{6.5in}{\centering
\begin{flushleft}

\vspace{2mm}
\hspace{5mm}
\textbf{\underline{Püstprisma täispindala}}

\vspace{2mm}
\hspace{5mm}
Püstprisma täispindala leidmiseks, võib leida iga püstprisma tahu pindala ja need kõik kokku liita.

\begin{equation}
\label{18_1}
S_{t}=S_{1}+S_{2}+S_{3}+...S_{n}
\end{equation}


\vspace{2mm}
\hspace{5mm}
Kuid kogu seda arvutamise protsessi võib mingil määral kiiremaks ja lihtsamaks teha. Nimelt teame,\\ \hspace{5mm} et kuna põhjad on püstprismal samad, siis piisab sellest, kui arvutame vaid ühe põhja pindala välja\\ \hspace{5mm} ja korrutame seda kahega.

\begin{equation}
\label{18_2}
S_{p}+S_{p} = 2\cdot S_{p}
\end{equation}

\vspace{2mm}
\hspace{5mm}
Kõikide külgtahkude pindalade leidmise asemel võib hoopis korrutada põhja ümbermõõdu $P$\\ \hspace{5mm} püstprisma kõrgusega $H$.

\begin{equation}
\label{18_3}
S_{k}=P_{\text{põhi}} \cdot H_{}
\end{equation}

\vspace{2mm}
\hspace{5mm}
Viimaks tuleb püstprisma täispindala leidmiseks lihtsalt külje pindala $S_{k}$ ja kaks põhjapindala $S_{p}$\\ \hspace{5mm} omavahel kokku liita.

\begin{equation}
\label{18_4}
\fbox{$S_{t}=S_{k}+2S_{p}$}
\end{equation}

\vspace{2mm}
\hspace{5mm}
NB! Tasub ka rõhutada, et põhja ümbermõõt $P$, põhjapindala $S_{p}$ sõltub sellest, milline kujund teil\\ \hspace{5mm} põhjas on. Juhul kui põhjaks on ruut, kolmnurk või mõni muu tuttav kujund, mis on meelest läinud,\\ \hspace{5mm} siis tasub meenutuseks uurida peatüki \ref{ruut}.

\vspace{5mm}
\hspace{5mm}
\textbf{\underline{Püstprisma ruumala}}

\vspace{2mm}
\hspace{5mm}
Püstprisma ruumala $V$ saab kätte korrutades põhja pindala $S_{p}$ prisma kõrgusega $H$.

\begin{equation}
\label{18_5}
\fbox{$V = S_{p} \cdot H$}
\end{equation}

\vspace{2mm}
\hspace{5mm}
Tihti esineb ka ülesandeid, kus palutakse leida mingi materjali mass $m$, kui on teada selle tihedus $\rho$ ja\\ \hspace{5mm} ruumala $V$. 

\vspace{2mm}
\hspace{5mm}
Kuna ruumala ühik on $m^{3}$ ning tiheduse ühik on $\dfrac{kg}{m^{3}}$, siis on ilusti näha, et kg (massi ühik) jääks meil\\ \hspace{5mm} alles vaid siis, kui me korrutaksime mõlemad suurused omavahel läbi (kuna kuupmeetrid taanduvad ära\\ \hspace{5mm} ning alles jääb vaid kilogramm).

\begin{equation}
\label{18_6}
m = \rho \cdot V
\end{equation}
\end{flushleft}
}}}
\end{center}



\newpage

\vspace{0.5cm}

\textbf{Märkmed}\\
\vspace{2mm}
\begin{mdframed}[style=graphpaper]
\vspace{21cm}
\end{mdframed}
