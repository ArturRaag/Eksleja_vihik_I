\begin{center}
\fbox{\fbox{\parbox{6.5in}{\centering
\begin{flushleft}

\vspace{2mm}
\hspace{5mm}
NB! Enne seda peatükki uurimist on soovituslik selgeks saada \ref{astmed} peatükk.

\vspace{5mm}
\hspace{5mm}
\textbf{\underline{Üksliikmete korrutamine}}

\vspace{2mm}
\hspace{5mm}
Üksliikmete korrutamisel korrutame kokku eraldi arvud ning seejärel ka samasugused tähed kasutades\\ \hspace{5mm} astmete reegleid. Kui tähed on erinevad, siis kirjutame need lihtsalt üksteise kõrvale tähestikulises\\ \hspace{5mm} järjekorras (meenutame, et kõrvuti olevate tähtede vahel on vaikimisi korrutusmärk).

\vspace{2mm}
\hspace{5mm}
Meenutuseks panema kirja astmete korrutamise üldise valemi:


\begin{equation}
\label{21_eq1}
a^{n} \cdot a^{m} = a^{n+m}
\end{equation}


\vspace{2mm}
\hspace{5mm}
\textbf{Näiteks:}

\vspace{2mm}
\hspace{5mm}
$-2x^{3}y \cdot 5xz^{7} =\left[ \begin{tabular}{c}
Korrutame esialgu\\
kokku vaid arvud\\
$-2 \cdot 5 = -10 $
\end{tabular} \right]=-10x^{3}y \cdot xz^{7} = \left[ \begin{tabular}{c}
Nüüd korrutame kokku\\
samasugused tähed\\
kasutades valemit \ref{21_eq1}.\\
NB! $a=a^{1}$.
\end{tabular} \right] = -10x^{4}yz^{7}$


\vspace{5mm}
\hspace{5mm}
\textbf{\underline{Üksliikmete jagamine}}

\vspace{2mm}
\hspace{5mm}
Jagamise puhul oleks pigem soovituslik kasutada murrujoone abi, kuna siis on ka selgemalt näha, mis\\ \hspace{5mm} ära taandub ning mis kuhu alles jääb.

\vspace{2mm}
\hspace{5mm}
Teadmiseks:

\vspace{2mm}
\hspace{5mm}
\begin{equation}
\label{21_eq2}
a^{n}=\underbrace{a\cdot a\cdot a \cdot a \cdot \cdot \cdot a}_\text{$n$ kordust}
\end{equation}


\vspace{2mm}
\hspace{5mm}
\textbf{Näiteks:}

\vspace{2mm} 
\hspace{5mm}
1) $3a^{4} : 7a^{9}=\dfrac{3a^{4}}{7a^{9}}= \left[\begin{tabular}{c}
Kirjutame numbrite\\
murru eraldi korrutisena.\\
(murdude omadus lubab)
\end{tabular} \right]=\dfrac{3}{7} \cdot \dfrac{a^{4}}{a^{9}}=\left[ \begin{tabular}{c}
Kirjutame tähed lahti\\
nagu valemis \ref{21_eq2}
\end{tabular} \right]=$\\
\vspace{5mm}
\hspace{5mm}
$=\dfrac{3}{7} \cdot \dfrac{a\cdot a \cdot a \cdot a}{a \cdot a \cdot a \cdot a \cdot a\cdot a\cdot a\cdot a\cdot a}= \left[ \begin{tabular}{c}
Taandame.
\end{tabular} \right]= \dfrac{3}{7} \cdot \dfrac{\cancel{a\cdot a \cdot a \cdot a}}{\cancel{a \cdot a \cdot a \cdot a} \cdot a\cdot a\cdot a\cdot a\cdot a}=$\\
\vspace{5mm}
\hspace{5mm}
$=\dfrac{3}{7} \cdot \dfrac{1}{1 \cdot a\cdot a\cdot a\cdot a\cdot a}= \left[\begin{tabular}{c}
Kirjutame nimetajas oleva\\
korrutise uuesti astmeks
\end{tabular} \right]=\dfrac{3}{7} \cdot \dfrac{1}{a^{5}}=\dfrac{3}{7 a^{5}}$

\vspace{2mm} 
\hspace{5mm}
Kui võrrelda avaldist pärast esimest võrdusmärki $\dfrac{3a^{4}}{7a^{9}}$ lõpptulemusega $\dfrac{3}{7 a^{5}}$, siis on näha, et tegelikult\\ \hspace{5mm} me eemaldasime lihtsalt mõlemalt poolt (nii nimetajast kui ka lugejast) nii palju $a$-sid, et kummaltki\\ \hspace{5mm} poolt need lõpuks otsa saaksid. Ehk ülevalt eemaldasime 4 $a$-d ja samapalju ka alt. Üleval said need\\ \hspace{5mm} lõpuks otsa, kuid alla jäi $9-4=5$ $a$-d. Sellist järeldust kasutame edaspidi pikkemate arvutuste\\ \hspace{5mm} vältimiseks.

\vspace{2mm}
\hspace{5mm}
\textbf{Näiteks:}\\
\hspace{5mm}
$9a^{4}k^{3}p^{8}:3a^{2}k^{3}p^{10}=\dfrac{9a^{4}k^{3}p^{8}}{3a^{2}k^{3}p^{10}}=\dfrac{3a^{2}}{p^{2}}$
\end{flushleft}
}}}
\end{center}


\pagebreak

\begin{center}
\fbox{\fbox{\parbox{6.5in}{\centering
\begin{flushleft}

\vspace{2mm}
\hspace{5mm}
\textbf{\underline{Negatiivne aste üksliikmes}}

\vspace{2mm}
\hspace{5mm}
Juhul kui kordaja aste on negatiivne (kas lugejas või nimetajas), siis vahetab kordaja enda kohta\\ \hspace{5mm} murrus (nt liigub nimetajast lugejasse või lugejast nimetajasse) ja märk muutub positiivseks.

\vspace{2mm}
\hspace{5mm}
\textbf{Näiteks:}

\vspace{2mm}
\hspace{5mm}
$\dfrac{4a^{2}b^{-3}}{xy^{-1}}=\dfrac{4a^{2}y}{b^{3}x}$


\vspace{5mm}
\hspace{5mm}
\textbf{\underline{Üksliikme astendamine}}

\vspace{2mm}
\hspace{5mm}
Üksliikme astendamisel astendatakse eraldi kõik kordajad vastavalt järgmisele valemile:
\begin{equation}
\label{21_eq3}
(ab)^{n}=a^{n}\cdot b^{n}
\end{equation}

\vspace{2mm}
\hspace{5mm}
Samuti kehtib ka järgmine valem:
\begin{equation}
\label{21_eq4}
(a^{m})^{n}=a^{mn}
\end{equation}

\vspace{2mm}
\hspace{5mm}
\textbf{Näiteks:}

\vspace{2mm}
\hspace{5mm}
$(-5xy^{4})^{3}=\left[ \begin{tabular}{c}
Arvestades valemit \ref{21_eq3}\\
tuleb meil ennem iga sulgudes\\
olev kordaja läbi astendada\\
arvuga 3.
\end{tabular} \right]=(-5)^{3}(x)^{3}(y^{4})^{3}=-125x^{3}(y^{4})^{3}=$\\
\vspace{5mm}
\hspace{5mm} $=\left[ \begin{tabular}{c}
On näha, et kui jõuame $y^{4}$-ni,\\
siis see kordaja on juba astmes.\\
Sellisel juhul rakendame valemit \ref{21_eq4}.
\end{tabular} \right]= -125x^{3}y^{4\cdot 3} =-125x^{3}y^{12} $
\end{flushleft}
}}}
\end{center}
\vspace{0.5cm}

\textbf{Märkmed}\\
\vspace{2mm}
\begin{mdframed}[style=graphpaper]
\vspace{8cm}
\end{mdframed}