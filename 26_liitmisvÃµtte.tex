\begin{center}
\fbox{\fbox{\parbox{6.5in}{\centering
\begin{flushleft}

\vspace{2mm}
\hspace{5mm}
\textbf{\underline{Võrrandisüsteemi lahendamine liitmisvõttega}}

\vspace{2mm}
\hspace{5mm}
Lahendame eelmises peatükkis olnud võrrandisüsteemi (\ref{25_eq3}) liitmisvõttega.

\vspace{2mm}
\hspace{5mm}
Kirjutame selle võrrandisüsteemi uuesti välja:


\begin{equation}
\label{26_eq1}
\begin{cases}
2x+5y=10\\
-3x+15y=1
\end{cases}
\end{equation}

\hspace{5mm}
Eesmärk on muuta kahe tundmatuga võrrandisüsteem hoopis ühe tundmatuga võrrandiks. Seda\\ \hspace{5mm} saaksime saavutada, kui saaksime mõlemale võrrandile kummagi tundmatu ($x$-i või $y$-i) jaoks\\ \hspace{5mm} vastandarvud.

\vspace{2mm}
\hspace{5mm}
Hetkel on näha, et kui korrutaksime esimese võrrandi $-3$ga, siis saaksime järgmise süsteemi:

\begin{equation}
\label{26_eq2}
\begin{cases}
2x+5y=10 \hspace{3mm} \bigg| \cdot (-3)\\
-3x+15y=1
\end{cases} 
\longrightarrow
\begin{cases}
-6x-15y=-30 \\
-3x+15y=1
\end{cases}
\end{equation}

\hspace{5mm}
On näha, et $y$-id on vastandarvud. Kui me nüüd võrrandi vasakud pooled ning paremad pooled\\ \hspace{5mm} omavahel kokku liidaksime, siis kaoksid $y$-id üldse ära, kuna $0y=0$. (Tasub rõhutada, et liitmine\\ \hspace{5mm} toimub veergude kaupa, ehk ülevalt alla liikudes).


\begin{center}
\begin{tabular}{c}
$\underline{+\begin{cases}
-6x-15y=-30 \\
-3x+15y=1
\end{cases}}$\\
$-9x+0=-29$
\end{tabular}
\end{center}

\hspace{5mm}
Jagame mõlemad võrduse pooled $-9$ga.

\[ -9x=-29 \hspace{3mm} \bigg| : (-9) \]

\[ \dfrac{-9x}{-9} = \dfrac{-29}{-9} \]

\[\dfrac{\cancel{-9} \cdot x}{\cancel{-9}} = \dfrac{29}{9}  \]

\[ x=\dfrac{29}{9} \approx 3.2 \]

\hspace{5mm}
Esimene tundmatu ($x$) on leitud, teise tundmatu ($y$) leidmise jaoks piisab, kui asendame saadud\\ \hspace{5mm} tundmatu $x=\dfrac{29}{9}$ kummagisse esialgsesse võrrandisse. Ehk valime süsteemist \ref{26_eq1} ühe võrrandi ja\\ \hspace{5mm} asendame sinna sisse leitud $x$-i.

\vspace{2mm}
\hspace{5mm}
Valime näiteks võrrandi $-3x+15y=1$ (NB! Valik ei ole oluline, lõpptulemust ei mõjuta!).\\ \hspace{5mm} Asendame $x=\dfrac{29}{9}$:

\[ -3 \cdot \dfrac{29}{9}+15y=1 \] 

\hspace{5mm}
Näide jätkub järgmisel lehel...


\end{flushleft}
}}}
\end{center}


\begin{center}
\fbox{\fbox{\parbox{6.5in}{\centering
\begin{flushleft}

\vspace{2mm}

\[ \dfrac{-3 \cdot 29}{9}+15y=1 \]

\hspace{5mm}
Avaldame $y$ - i.

\[ 15y=1+\dfrac{3 \cdot 29}{9} \hspace{3mm} \bigg| : 15\]

\[ \dfrac{15y}{15}=\dfrac{1}{15} + \dfrac{3 \cdot 29}{9 \cdot 15} \]

\[ \dfrac{\cancel{15} \cdot y}{\cancel{15}}=\dfrac{1}{15} + \dfrac{\cancel{3} \cdot 29}{\cancel{9} \cdot 15} \hspace{3mm} \longrightarrow y = \dfrac{1}{15}+\dfrac{1 \cdot 29}{3 \cdot 15} \]

\[ y = \dfrac{1}{15} + \dfrac{29}{45} = \left[ \text{Laiendame} \right] = \dfrac{1}{15}^{(3} + \dfrac{29}{45}^{(1} = \dfrac{3}{45}+ \dfrac{29}{45}=\dfrac{3+29}{45}=\dfrac{32}{45} \]

\[ y= \dfrac{32}{45} \approx 0.71 \]

\hspace{5mm}
Seega võrrandisüsteemi \ref{26_eq1} täpsed lahendid on:

\[ \begin{cases}
x=\dfrac{29}{9}\\
\\
y=\dfrac{32}{45}
\end{cases} \]
\end{flushleft}
}}}
\end{center}

\vspace{0.5cm}

\textbf{Märkmed}\\
\vspace{2mm}
\begin{mdframed}[style=graphpaper]
\vspace{9cm}
\end{mdframed}