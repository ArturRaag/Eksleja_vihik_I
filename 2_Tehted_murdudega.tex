\begin{center}
\fbox{\fbox{\parbox{6.5in}{\centering
\begin{flushleft}
\vspace{5mm}
\hspace{5mm} \textbf{\underline{Ühenimelised murrud:}}\\
\vspace{5mm}
\hspace{5mm} 1) Liitmine: $\dfrac{a}{b}+\dfrac{c}{b}=\dfrac{a+c}{b}$,\hspace{30mm} Näiteks: $\dfrac{2}{5}+\dfrac{1}{5}=\dfrac{2+1}{5}=\dfrac{3}{5}$\\
\vspace{5mm}
\hspace{5mm} 2) Lahutamine: $\dfrac{a}{b}-\dfrac{c}{b} = \dfrac{a-c}{b}$,\hspace{25mm} Näiteks: $\dfrac{5}{7}-\dfrac{2}{7}=\dfrac{5-2}{7}=\dfrac{3}{7}$\\
\vspace{5mm}
\hspace{5mm} \fbox{Üldisemalt: $\dfrac{a}{b} \pm \dfrac{c}{b}=\dfrac{a \pm c}{b}$} ,\\
\vspace{12mm}
\hspace{5mm} \textbf{\underline{Erinimelised murrud:}}\hspace{2mm} \\
\vspace{5mm}
\hspace{5mm} 1) Liitmine: $\dfrac{a}{b}+\dfrac{x}{y}={\dfrac{a}{b}}^{(y}+{\dfrac{x}{y}}^{(b}=\dfrac{a \cdot y}{b \cdot y}+\dfrac{x \cdot b}{y \cdot b}=
\bigg[ \begin{tabular}{c}
Nüüd on nagu\\
ühenimelised
\end{tabular} \bigg]=\dfrac{a\cdot y + x\cdot b}{b \cdot y}$,\\
\vspace{5mm}
\hspace{5mm} 2) Lahutamine: $\dfrac{a}{b}-\dfrac{x}{y}={\dfrac{a}{b}}^{(y}-{\dfrac{x}{y}}^{(b}=\dfrac{a \cdot y}{b \cdot y}-\dfrac{x \cdot b}{y \cdot b}=
\bigg[ \begin{tabular}{c}
Nüüd on nagu\\
ühenimelised
\end{tabular} \bigg]=\dfrac{a\cdot y - x\cdot b}{b \cdot y}$,\\
\vspace{5mm}
\hspace{5mm} \fbox{ Üldisemalt: $\dfrac{a}{b} \pm \dfrac{x}{y}=\dfrac{a\cdot y \pm x\cdot b}{b \cdot y}$} ,\\
\vspace{5mm}
\hspace{5mm} Liitmise näide: $\dfrac{2}{3}+\dfrac{1}{4}={\dfrac{2}{3}}^{(4}+\dfrac{1}{4}^{(3}=\dfrac{2 \cdot 4}{3 \cdot 4}+\dfrac{1 \cdot 3}{4 \cdot 3}=\dfrac{8}{12}+\dfrac{3}{12}=\dfrac{8+3}{12}=\dfrac{11}{12}$\\
\vspace{5mm}
\hspace{5mm} Lahutamise näide: $\dfrac{6}{7}-\dfrac{1}{2} = {\dfrac{6}{7}}^{(2}-{\dfrac{1}{2}}^{(7}= \dfrac{6 \cdot 2}{7 \cdot 2}-\dfrac{1 \cdot 7}{2 \cdot 7}=\dfrac{12}{14}-\dfrac{7}{14}=\dfrac{12-7}{14}=\dfrac{5}{14}$\\
\vspace{5mm}
\hspace{5mm} \textbf{\underline{Murdude korrutamine:}} \fbox{$\dfrac{a}{b} \cdot \dfrac{x}{y}= \dfrac{a \cdot x}{b \cdot y}$}\\
\vspace{5mm}
\hspace{5mm} Näiteks: $\dfrac{2}{5} \cdot \dfrac{3}{4} = \dfrac{2 \cdot 3}{5 \cdot 4 } = \dfrac{6}{20}$\\
\vspace{5mm}
\hspace{5mm} \textbf{\underline{Murdude jagamine:}} \fbox{$\dfrac{a}{b} : \dfrac{x}{y} = \dfrac{a}{b} \cdot \dfrac{y}{x}=\dfrac{a \cdot y}{b \cdot x}$}\\
\vspace{5mm}
\hspace{5mm} Näiteks: $\dfrac{4}{5}:\dfrac{2}{3}=\dfrac{4}{5} \cdot \dfrac{3}{2}=\dfrac{4 \cdot 3}{5 \cdot 2}= \dfrac{12}{10}= \Bigg[ \begin{tabular}{c}
10 mahub 12-ne\\
sisse vaid 1 korra,\\
jääk lugejas on 2
\end{tabular} \Bigg] = 1\dfrac{2}{10}= \Bigg[ \begin{tabular}{c}
2 ja 10\\
jaguvad 2-ga,\\
saame taandada.
\end{tabular} \Bigg]=1\dfrac{1}{5}$
\end{flushleft}
}}}
\end{center}

\newpage

\textbf{Autori poolsed kommentaarid}\\

1) Kui meil on ühenimelised murrud (mõlema murrujoone all on identsed arvud), siis murdude liitmise/lahutamise korral, liidame/lahutame vaid murru lugejas olevad arvud. Nimetaja jääb samaks.\\

2) Kui meil on tegemist erinimeliste murdudega (murrujoonte all olevad numbrid on erinevad), siis tuleb need teha ühenimelisteks. Seda saab saavutada murde laiendades. Kõige kindlam viis oleks kasutada "ristlaiendust", aga loomulikult kui piisab vaid ühe murru laiendamisest siis võib ristlaiendust vältida ja laiendada vaid ühte kahest murrust. Peaasi, et mõlemal murrul on ühesugused nimetajad (murrujoone all olevad numbrid).\\

\vspace{1.5cm}

\textbf{Märkmed}\\
\vspace{2mm}
\begin{mdframed}[style=graphpaper]
\vspace{15cm}
\end{mdframed}
