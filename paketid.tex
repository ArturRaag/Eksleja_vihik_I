
%%%%%%%%%%%%%%%%% PACKAGES %%%%%%%%%%%%%%%%%%%%
\usepackage[utf8]{inputenc}
\usepackage[T1]{fontenc}
%\usepackage{fontspec}
\usepackage{titlesec}
\usepackage{xcolor} % for black background white text
\titleformat{\chapter}[display]
  {\normalfont\bfseries\centering}{}{0pt}{\huge}  
\titlespacing*{\chapter}{0pt}{0pt}{10pt}

\usepackage[estonian]{babel}
\usepackage{gensymb}
\usepackage{graphicx}
\usepackage{pgfplots}
\usepackage{url}
\usepackage{cancel}
\usepackage{amsmath} % for \dfrac macro
\usepackage{subfig}
\usepackage[colorinlistoftodos]{todonotes}
\usepackage[colorlinks=true, allcolors=blue]{hyperref}
\usepackage{hyperref}
\usepackage{wrapfig} % to wrap text around figure
\newcommand\ddfrac[2]{{\displaystyle\frac{\displaystyle #1}{\displaystyle #2}}}


\usepackage{yfonts} % For gothic font

\usepackage[framemethod=tikz]{mdframed} % RUUDUSTIKU JAOKS

\usetikzlibrary{backgrounds}

\mdfdefinestyle{graphpaper}{%
    apptotikzsetting={\tikzset{mdfbackground/.style={}}},
    singleextra={%
        \scoped[on background layer,yshift=\mdfboundingboxheight]{\draw[step=5mm, line width=0.2mm, black!20!white] (0,0) grid (\mdfboundingboxwidth,-\mdfboundingboxheight);}
    },
}


\usepackage{titlepic} % For cover page
\usepackage{pdfpages}

\usepackage{tikz} % Igasuguste kujundite joonistamise jaoks
\usetikzlibrary{intersections, calc, angles} % nurkade joonestamise jaoks



%%%%%%%%%%%%%%%%%%%%%%%%%%%%%%%%%%%%%%%%%%%%%%%%%%%%%%%%%%%%%%%%%%%%%%%%%%%%%%%%%%%%%%
\usepackage{xparse,array} %korrutustabeli jaoks
\ExplSyntaxOn

\NewDocumentCommand{\multiplicationtable}{O{2em}m}
 {
  \azetina_multiplicationtable:nn { #1 } { #2 }
 }

\tl_new:N \l__azetina_multiplicationtable_tl

\cs_new_protected:Nn \azetina_multiplicationtable:nn
 {
  \tl_set:Nn \l__azetina_multiplicationtable_tl { $\times$ }
  \int_step_inline:nn { #2 }
   {
    \tl_put_right:Nn \l__azetina_multiplicationtable_tl { & ##1 }
   }
  \tl_put_right:Nn \l__azetina_multiplicationtable_tl { \\ \hline }
  \int_step_inline:nn { #2 }
   {
    \tl_put_right:Nn \l__azetina_multiplicationtable_tl { ##1 }
    \int_step_inline:nn { #2 }
     {
      \tl_put_right:Nx \l__azetina_multiplicationtable_tl
       {
        & \int_to_arabic:n { ##1*####1 }
       }
     }
    \tl_put_right:Nn \l__azetina_multiplicationtable_tl { \\ }
   }
  \begin{tabular}{ @{} r |@{}  *{#2}{w{r}{#1}@{}} }
  \tl_use:N \l__azetina_multiplicationtable_tl
  \end{tabular}
 }

\ExplSyntaxOff
%%%%%%%%%%%%%%%%%%%%%%%%%%%%%%%%%%%%%%%%%%%%%%%%%%%%%%%%%%%%%%%%%%%%%%%%%%%%%%%%%%%%%%%



% kaarte joonestamiseks:
\def\centerarc[#1](#2)(#3:#4:#5)% Syntax: [draw options] (center) (initial angle:final angle:radius)
    { \draw[#1] ($(#2)+({#5*cos(#3)},{#5*sin(#3)})$) arc (#3:#4:#5); }

\usepackage{tikz,tkz-euclide} % võrdsete haarade tähistamiseks

