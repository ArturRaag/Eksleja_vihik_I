\begin{center}
\fbox{\fbox{\parbox{6.5in}{\centering

\begin{flushleft}
\vspace{5mm}
\hspace{5mm} \textbf{\underline{Arv mingisuguses astmes:}}\\
\vspace{5mm}
\hspace{5mm} $a^{n}=\underbrace{a\cdot a\cdot a \cdot a \cdot \cdot \cdot a}_\text{$n$ kordust}$ 
\hspace{10mm} Näiteks: $4^{5}=4\cdot 4 \cdot 4 \cdot 4 \cdot 4 = 1024$\\
\vspace{5mm}
\hspace{5mm} \textbf{Juhul, kui on tegemist murruga}\\
\vspace{5mm}
\hspace{5mm} $\bigg(\dfrac{a}{b}\bigg)^{n}=\underbrace{\dfrac{a}{b} \cdot \dfrac{a}{b}\cdot \cdot \cdot \dfrac{a}{b}}_\text{$n$ kordust}=\underbrace{\dfrac{a \cdot a \cdot \cdot \cdot a}{b \cdot b \cdot \cdot \cdot b}}_\text{\begin{tabular}{c} 
lugejas ja\\
nimetajas \\ 
$n$ kordust
\end{tabular}}=\dfrac{a^{n}}{b^{n}}$
\hspace{10mm} Näiteks: $\bigg(\dfrac{2}{5}\bigg)^{3}=\dfrac{2^{3}}{5^{3}}=\dfrac{8}{125}$\\
\vspace{5mm}
\hspace{5mm} \textbf{Juhul, kui me astendame negatiivset arvu, siis}\\
\vspace{5mm}
\hspace{5mm} $(-a)^{n}=$
$\begin{cases}
\mbox{vastus on \textbf{POSITIIVNE}, kui $n$ on paarisarv (2, 4, 6, 8...jne)}\\
\mbox{vastus on \textbf{NEGATIIVNE}, kui $n$ on paaritu arv (1, 3, 5, 7, 9...jne)}
\end{cases}$\\
\vspace{5mm}
\hspace{5mm} Mõned näited:\\
\vspace{2mm}
\hspace{5mm} a) $(-5)^{4}= (-5) \cdot (-5) \cdot (-5) \cdot (-5)= 625$\\
\vspace{2mm}
\hspace{5mm} b) $(-5)^{3}=(-5) \cdot (-5) \cdot (-5)= -125$ (NB! pane tähele, et vastus on siin miinusega, kuna aste paaritu!)

\vspace{5mm}
\hspace{5mm} \textbf{\underline{Astmete korrutamine ning jagamine:}}\\
\vspace{5mm}
\hspace{5mm} \textbf{Korrutamine:} $a^{n} \cdot a^{m} = a^{n+m}$\\
\vspace{2mm}
\hspace{5mm} Näiteks: $5^{10} \cdot 5^{12} = 5^{10+12}=5^{22}$\\
\vspace{5mm}
\hspace{5mm} \textbf{Jagamine:} $a^{n} : a^{m} = a^{n-m}$\\
\vspace{2mm}
\hspace{5mm} Näiteks: $8^{23} : 8^{12}= 8^{23-12}=8^{11}$\\
\vspace{5mm}
\hspace{5mm} \textbf{\underline{Korrutise astendamine}}\\
\vspace{5mm}
\hspace{5mm} $(a \cdot b)^{n}= \underbrace{(a \cdot b) \cdot (a \cdot b) \cdot \cdot \cdot (a \cdot b)}_\text{$n$ kordust}=a \cdot b \cdot a \cdot b \cdot \cdot \cdot a \cdot b= \underbrace{a\cdot a \cdot \cdot \cdot a}_\text{$n$ kordust} \cdot \underbrace{b \cdot b \cdot \cdot \cdot b}_\text{$n$ kordust}=a^{n}\cdot b^{n}$\\
\vspace{2mm}
\hspace{5mm} Näiteks: $(2\cdot 5)^{3}=2^{3}\cdot 5^{3}=8 \cdot 125 = 1000$\\
\vspace{5mm}

\end{flushleft} }}}
\end{center}

\newpage

\begin{center}
\fbox{\fbox{\parbox{6.5in}{\centering

\begin{flushleft}

\hspace{5mm} \textbf{\underline{Astmete astendamine}}\\
\vspace{5mm}
\hspace{5mm} $(a^{n})^{k}=\underbrace{a^{n} \cdot a^{n} \cdot a^{n} \cdot \cdot \cdot a^{n}}_\text{$k$ kordust}=\overbrace{a^{n+n+n+...+n}}^\text{$k$ liiget}=a^{(k \cdot n)}$\\
\vspace{5mm}
\hspace{5mm} Näiteks: $(3^{2})^{5}=3^{2\cdot 5}=3^{10}$


\end{flushleft} }}}
\end{center}

\textbf{Märkmed}\\
\vspace{2mm}
\begin{mdframed}[style=graphpaper]
\vspace{14cm}
\end{mdframed}