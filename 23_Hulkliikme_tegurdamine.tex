\begin{center}
\fbox{\fbox{\parbox{6.5in}{\centering
\begin{flushleft}

\vspace{2mm}
\hspace{5mm}
\textbf{\underline{Hulkliikme tegurdamine (sulgude ette toomine)}}

\vspace{2mm}
\hspace{5mm}
Eelnevalt korrutasime hulkliikmeid üksliikmetega läbi, et saada uut hulkliiget. Kuid vahest on kasulik\\ \hspace{5mm} ka vastupidine protsess (põhiliselt avaldise lihtsustamise eesmärgil). Ühise teguri hulkliikme sulgude\\ \hspace{5mm} ette toomine ongi hulkliikme tegurdamine. Ühise teguri all mõeldakse siin arvu, millega kõik hulkliikme\\ \hspace{5mm} üksliikmed jaguvad (see võib olla nii number, täht, tähed või isegi kõik korraga). Oskus hinnata, millal\\ \hspace{5mm} tasub avaldist lihtsustamise eesmärgil tegurdada, kujuneb välja vaid harjutamisega (nagu kõik muugi),\\ \hspace{5mm} mille tõttu on rangelt soovituslik võimalikult palju harjutusi lahendada.


\vspace{2mm}
\hspace{5mm}
\textbf{Näited:}

\vspace{2mm}
\hspace{5mm}
1) $14x-21y=\left[\begin{tabular}{c}
Nii $14x$, kui ka $21y$ olid saadud\\
7-ga läbikorrutades (mõlemad\\
jaguvad 7-ga). Toome $7$ sulgude\\
ette ning esialgse avaldise jagame\\
7-ga läbi.
\end{tabular} \right]=7\left( \dfrac{14x}{7}-\dfrac{21y}{7} \right)=7(2x-3y)$

\vspace{5mm}
\hspace{5mm}
2) $4x^{3}+2xy=\left[ \begin{tabular}{c}
Nii $4x^{3}$ kui ka $2xy$ olid saadud\\
$2x$-iga läbikorrutades (mõlemad\\
jaguvad $2x$-iga). Toome $2x$ sulgude\\
ette ning esialgse avaldise jagame\\
$2x$-iga läbi.
\end{tabular} \right]=2x\left( \dfrac{4x^{3}}{2x} + \dfrac{2xy}{2x} \right)=2x(2x^{2}+y)$


\vspace{5mm}
\hspace{5mm}
\textbf{\underline{Hulkliikme tegurdamine (rühmitamisvõte)}}

\vspace{2mm}
\hspace{5mm}
Mõnikord saame teisendada üksliikmete summa hulkliikmete korrutiseks. Korrutiseks teisendamist\\ \hspace{5mm} nimetataksegi tegurdamist. Teine tegurdamise meetod on rühmitamisvõte.

\vspace{2mm}
\hspace{5mm}
\textbf{Näited:}

\vspace{2mm}
\hspace{5mm}
1) $\underbrace{7x+xy}_\text{1 rühm}-\underbrace{2y-14}_\text{2 rühm}=\left[ \begin{tabular}{c}
1 rühmal on ühiseks teguriks $x$.\\
2 rühmal on ühiseks teguriks $-2$.\\
Toome iga rühma ühise teguri sulgude\\
ette.
\end{tabular}  \right]= x(7+y)-2(y+7)=$

\vspace{5mm}
\hspace{5mm}
$= \left[ \begin{tabular}{c}
Muudame sulgudes\\
järjekorra, kuna\\
(7+y)=(y+7)
\end{tabular} \right] = x(7+y)-2(7+y) = \left[ \begin{tabular}{c}
Sulgude ees olevad tegurid $x$ ja $-2$\\
moodustavad esimese hulkliikme ning\\
sulgudes olev hulkliige moodustab teise\\
hulkliikme. Need ongi meie kaks tegurit.
\end{tabular} \right]=$

\vspace{5mm}
\hspace{5mm}
$=(x-2)(7+y)$

\vspace{10mm}
\hspace{5mm}
2) $12x^{2}-15x+8x-10=\underbrace{12x^{2}-15x}_\text{1 rühm}+\underbrace{8x-10}_\text{2 rühm}=3x(4x-5)+2(4x-5)=(3x+2)(4x-5)$

\end{flushleft}
}}}
\end{center}

\pagebreak
\vspace{0.5cm}

\textbf{Märkmed}\\
\vspace{2mm}
\begin{mdframed}[style=graphpaper]
\vspace{19cm}
\end{mdframed}