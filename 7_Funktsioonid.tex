\begin{center}
\fbox{\fbox{\parbox{6.5in}{\centering
\begin{flushleft}
\vspace{5mm}
\hspace{5mm}
\textbf{\underline{Punkt A koordinaatidega (x;y)}} \\
\vspace{5mm}
\hspace{5mm}
Tähistame järgmiselt: $A(x;y)$.\\
\hspace{5mm}
$x$ - kui palju liigume nullpunktist paremale või vasakule. Kui $x$ on negatiivne ($x<0$), siis liigume\\ \hspace{5mm} vasakule. Kui $x$ on positiivne ($x>0$), liigume paremale.\\
\vspace{2mm}
\hspace{5mm}
$y$ - kui palju liigume (pärast x-i saabumist) ülesse või alla. Kui $y$ on negatiivne $(y<0)$, siis liigume\\ \hspace{5mm} alla. Kui $y$ on positiivne $(y>0)$, liigume ülesse.\\
\vspace{5mm}
\hspace{5mm}
Näiteks joonisel \ref{img1} on esitatud punktid: A(4; 6), B(-4; 6), C(-4; -6) ja D(4; -6).
\begin{center}
\begin{minipage}{7cm}
  \includegraphics[width=6.8cm]{Figure_1.png}
  \captionof{figure}{Punktid A, B, C ja D.}
  \label{img1}
\end{minipage}
\hspace{.05\linewidth}
\begin{minipage}{.45\linewidth}
  \includegraphics[width=6.8cm]{Figure_2.png}
  \captionof{figure}{$y=-3x+2$}
  \label{img2}
\end{minipage}
\end{center}

\hspace{5mm}
\textbf{\underline{Lineaarfunktsioon $y=ax+b$ ja selle graafik}}\\
\vspace{5mm}
\hspace{5mm}
Lineaarfunktsioonis nimetatakse kordajat $a$ tõusuks. See määrab meil funktsiooni muutumise kiiruse.\\ \hspace{5mm} Mida suurem $a$, seda kiiremini toimuvad muutused. Kui $a$ on positiivne, siis toimub $y$ kasvamine. \\ \hspace{5mm} Kui $a$ on negatiivne, siis toimub $y$  kahanemine/langemine. Kui $a=0$, siis on sirge paralleelne x-teljega,\\ \hspace{5mm} kuid selle kaugus x-teljest sõltub liikmest $b$ (ja seega $y=b$, mis tähendab et $y$ ei sõltu $x$-ist).\\
\vspace{2mm}
\hspace{5mm}
Lineaarfunktsiooni liiget $b$ nimetame vabaliikmeks. Kui $b=0$, (ehk kui $b$ puudub), siis läbib lineaar-\\ \hspace{5mm} funktsiooni sirge xy-telgede nullpunkte. Kui $b$ on positiivne, siis nihutame sirget ülespoole (mööda\\ \hspace{5mm} y-telge) $b$ ühiku võrra. Kui $b$ on negatiivne, siis nihutame sirget samuti mööda y-telge $b$ ühiku võrra\\ \hspace{5mm} allapoole.\\
\hspace{5mm} Lineaarfunktsiooni graafiku joonestamiseks piisab, kui valime kaks \textbf{suvalist} $x$-i väärtust ja arvutame\\ \hspace{5mm} valitud $x$-idega välja vastavad $y$-i väärtused. Tavaliselt tehakse seda tabeli kujul.\\
\vspace{5mm}
\hspace{5mm}
\textbf{Näide:} Joonestame lineaarfunktsiooni $y=-3x+2$ graafiku.\\
\vspace{5mm}
\hspace{5mm}
Valime näiteks $x_{1}=-1$ ja $x_{2}=1$. Arvutame välja neile vastavad $y$ väärtused:\\
\vspace{2mm}
\hspace{5mm}
$y_{1}=-3\cdot x_{1}+2 \hspace{3mm} \longrightarrow \hspace{3mm} y_{1}=-3\cdot (-1) + 2\hspace{3mm} \longrightarrow \hspace{3mm}  y_{1}=5$

\vspace{2mm}
\hspace{5mm}
$y_{2}=-3\cdot x_{2}+2 \hspace{3mm} \longrightarrow \hspace{3mm} y_{2}=-3\cdot 1 + 2\hspace{3mm} \longrightarrow \hspace{3mm}  y_{2}=-1$

\vspace{2mm}
\hspace{5mm} Näide jätkub järgmisel lehel...

\end{flushleft} }}}
\end{center}


\newpage


\begin{center}
\fbox{\fbox{\parbox{6.5in}{\centering
\begin{flushleft}
\vspace{2mm}
\hspace{5mm} Saame järgmise tabeli:\\
\vspace{5mm}
\hspace{15mm}
    \begin{tabular}{c|c|c}
         x & -1 & 1 \\
        \hline
         y & 5 & -1\\
    \end{tabular}\\
\vspace{2mm}
\hspace{5mm}
Kanname nüüd punktid ($x_{1};y_{1}$) ja ($x_{2};y_{2}$) ehk ($-1;5$) ja ($1;-1$) graafikule (Joonis \ref{img2}), ning seejärel\\ \hspace{5mm} tõmbame sirge, mis läbib mõlemat punkti. Sellega on graafik joonestatud.\\
\vspace{2mm}
\hspace{5mm}
Nagu eelnevalt mainitud, toimub meil negatiivse tõusu tõttu $x$-i väärtuste suurenemisel hoopis $y$\\ \hspace{5mm} väärtuse kahanemine. Samuti on näha, et meie sirge on nullpunktist nihutatud 2 ühiku võrra ülesse\\ \hspace{5mm} poole, kuna vabaliige on meil $b=2$.

\end{flushleft} }}}
\end{center}

\vspace{1cm}
\textbf{Märkmed}\\
\vspace{2mm}
\begin{mdframed}[style=graphpaper]
\vspace{14cm}
\end{mdframed}