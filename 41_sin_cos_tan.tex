\begin{center}
\fbox{\fbox{\parbox{6.5in}{\centering
\begin{flushleft}

\vspace{2mm}
\hspace{5mm}
\textbf{\underline{Trigonomeetria}}


\vspace{2mm}
\hspace{5mm}
\begin{tikzpicture}[scale=0.5]

\coordinate (A) at (0,0);
\coordinate (B) at (8,0);
\coordinate (C) at (8,4);

\draw[line width= 0.4mm] (A)--(B)--(C)--cycle;

\path (A)--(B) coordinate[pos=0.5](a);
\path (B)--(C) coordinate[pos=0.5](b);
\path (C)--(A) coordinate[pos=0.5](c);

\node[below] at (a){$a$};
\node[right] at (b){$b$};
\node[above] at (c){$c$};

\node[left] at (A){$A$};
\node[right] at (B){$B$};
\node[above] at (C){$C$};

\pic [draw, angle radius = 5mm ] {angle=C--B--A} node at (7.6,0.4){$\cdot$};
\pic [draw, angle radius= 10mm] {angle=B--A--C} node at (1.2,0.3){$\alpha$};
\pic [draw, angle radius = 7mm] {angle = A--C--B} node at (7.6,3.2){$\beta$};

\node[text width=9 cm] at (20,2){Nurga \textbf{siinus} on võrdne selle nurga \textbf{vastaskaateti} ja \textbf{hüpotenuusi} suhtega.

\begin{equation}
\label{41_eq1}
\boxed{sin(\alpha)= \dfrac{vk}{hu}}
\end{equation}};
\end{tikzpicture}

\vspace{2mm}
\hspace{5mm}
Nurga \textbf{koosinus} on võrdne selle nurga \textbf{lähiskaateti} ja \textbf{hüpotenuusi} suhtega.

\begin{equation}
\label{41_eq2}
\boxed{cos(\alpha)=\dfrac{lk}{hu}}
\end{equation}

\vspace{2mm}
\hspace{5mm}
Nurga \textbf{tangens} on võrdne selle nurga \textbf{vastaskaateti} ja \textbf{lähiskaateti} suhtega.

\begin{equation}
\label{41_eq3}
\boxed{tan(\alpha)=\dfrac{vk}{lk}}
\end{equation}

\vspace{2mm}
\hspace{5mm}
Meie joonise järgi:

\begin{equation}
\label{41_eq4}
\begin{tabular}{l}
$sin(\alpha)=\dfrac{b}{c}$ \hspace{7mm} $cos(\alpha)=\dfrac{a}{c}$ \hspace{7mm} $tan(\alpha)=\dfrac{b}{a}$\\
\\
$sin(\beta)=\dfrac{a}{c}$ \hspace{7mm} $cos(\beta)=\dfrac{b}{c}$ \hspace{7mm} $tan(\beta)=\dfrac{a}{b}$
\end{tabular}
\end{equation} 

\vspace{2mm}
\hspace{5mm}
Nurga väärtuste leidmiseks tulevad kasuks ka trigonomeetrilised pöördfunktsioonid: arkusiinus,\\ \hspace{5mm} arkuskoosinus, arkustangens.

\vspace{2mm}
\hspace{5mm}
Kalkulaatoritel või arvutitel tähistatakse neid järgmiselt: $arcsin(x)$, $arccos(x)$, $arctan(x)$ või $sin^{-1}(x)$,\\ \hspace{5mm} $cos^{-1}(x)$, $tan^{-1}(x)$. Kus $x$ on suvaline teravnurk.

\vspace{2mm}
\hspace{5mm}
\textbf{Näiteks:} Leiame nurga $\alpha$ väärtuse, kui: $sin(\alpha)=\dfrac{6}{12}$

\hspace{5mm}

\[ sin(\alpha)=\dfrac{6}{12}=\dfrac{1}{2} \] 

\hspace{5mm}
Paneme mõlemad võrdusepooled arkusiinuse alla.

\[ sin(\alpha)=\dfrac{1}{2} \hspace{2mm} \Big| sin^{-1}() \]

\[ sin^{-1}(sin(\alpha))= sin^{-1} \left( \dfrac{1}{2} \right) \hspace{3mm} \longrightarrow \alpha = 30 \]

\end{flushleft}
}}}
\end{center}




\begin{center}
\fbox{\fbox{\parbox{6.5in}{\centering
\begin{flushleft}


\vspace{2mm}
\hspace{5mm}
\textbf{\underline{Siinus ja koosinusteoreemid*}}

\vspace{2mm}
\hspace{5mm}
\begin{tikzpicture}[scale=0.4]
\coordinate (A) at (0,0);
\coordinate (B) at (8,0);
\coordinate (C) at (10,4);

\draw[line width=0.4mm] (A)--(B)--(C)--cycle;

\path (A)--(B) coordinate[pos=0.5](c);
\path (B)--(C) coordinate[pos=0.5](a);
\path (C)--(A) coordinate[pos=0.5](b);

\node[right] at (a){$a$};
\node[above] at (b){$b$};
\node[below] at (c){$c$};

\pic [draw, angle radius = 11mm ] {angle=B--A--C} node at (1.9,0.4){$\alpha$};
\pic [draw, angle radius = 8mm ] {angle=A--C--B} node at (9,3){$\gamma$};
\pic [draw, angle radius = 5mm ] {angle=C--B--A} node at (7.7,0.5){$\beta$};

\node[text width=9cm] at (-15,2){Pythagorase teoreem \ref{40_eq1} kehtib vaid täisnurkse kolmnurga puhul. Selleks, et lahendada kolmnurki, mis ei ole täisnurksed, saab kasutada siinus- ja koosinusteoreeme.
};
\end{tikzpicture}

\vspace{2mm}
\hspace{5mm}
\textbf{Siinusteoreem}

\begin{equation}
\label{41_eq5}
\boxed{\dfrac{a}{sin(\alpha)}=\dfrac{b}{sin(\beta)}=\dfrac{c}{sin(\gamma)}}
\end{equation}


\vspace{5mm}
\hspace{5mm}
\textbf{Koosinusteoreem}

\begin{equation}
\label{41_eq6}
\boxed{
\begin{tabular}{l}
$a^{2}=b^{2}+c^{2}-2bc \cdot cos(\alpha)$\\
\\
$b^{2}=a^{2}+c^{2}-2ac \cdot cos(\beta)$\\
\\
$c^{2}=a^{2}+b^{2}-2ab \cdot cos(\gamma)$\\
\end{tabular}
}
\end{equation}

\end{flushleft}
}}}
\end{center}

\vspace{0.5cm}

\textbf{Märkmed}\\
\vspace{2mm}
\begin{mdframed}[style=graphpaper]
\vspace{3cm}
\end{mdframed}