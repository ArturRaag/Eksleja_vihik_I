\begin{center}
\fbox{\fbox{\parbox{6.5in}{\centering
\begin{flushleft}

\hspace{5mm}
\textbf{\underline{Üksliikmed}}

\vspace{2mm}
\hspace{5mm}
Üksliikmed on avaldised, mis on saadud arvuliste ja täheliste tegurite korrutamise teel. Üldiselt, kui\\ \hspace{5mm} tähed on omavahel läbi korrutatud, siis korrutusmärki nende vahele ei kirjutata.

\vspace{2mm}
\hspace{5mm}
\textbf{Näited:}

\vspace{2mm}
\hspace{5mm}
1) $2\cdot x \cdot y \cdot 15 \cdot a \cdot b \longrightarrow 30xyab$

\vspace{2mm}
\hspace{5mm}
2) $x \cdot x \cdot y \cdot 4 \cdot y \longrightarrow 4xxyy$

\vspace{2mm}
\hspace{5mm}
3) $2$

\vspace{2mm}
\hspace{5mm}
4) $k$

\vspace{5mm}
\hspace{5mm}
\textbf{\underline{Üksliikme normaalkuju}}

\vspace{2mm}
\hspace{5mm}
Mugavuse mõttes, tuleks alati kõik üksliikmed viia normaalkujule. Selle jaoks tuleb:\\

\vspace{2mm}
\hspace{5mm}
1) üksliikmes olevad numbrid omavahel läbi korrutada ja panna saadud korrutis tähtede ette\\ \hspace{5mm} (nim. üksliikme kordajaks).\\ \hspace{5mm} NB! kui tähtede ees on $1$, siis jäetakse see kirjutamata! Kui $-1$, siis pannakse vaid $-$.

\vspace{2mm}
\hspace{5mm}
2) tähed korrutada omavahel läbi, kasutades astmete korrutamise reegleid, (meenutuseks uuri\\ \hspace{5mm} peatüki \ref{astmed} astmete korrutamise osa) ja paigutada need ümber tähestikulises järjekorras.

\vspace{2mm}
\hspace{5mm}
\textbf{Näiteks:}

\vspace{2mm}
\hspace{5mm}
Viime järgmised üksliikmed normaalkujule.

\vspace{2mm}
\hspace{5mm}
1) $5 \cdot z \cdot \ b \cdot z \cdot \left( -\dfrac{1}{5} \right)  \longrightarrow -bz^{2}$

\vspace{2mm}
\hspace{5mm}
2) $x \cdot y \cdot 15 \cdot x \longrightarrow 15x^{2}y$

\vspace{2mm}
\hspace{5mm}
3) $4 \cdot k \cdot f \cdot 2 \cdot h \cdot 3 \longrightarrow 24fhk$


\vspace{5mm}
\hspace{5mm}
\textbf{\underline{Sarnaste liikmete koondamine}}

\vspace{2mm}
\hspace{5mm}
Üksliikmed on sarnased, kui nad erinevad vaid nende ees oleva numbri (kordaja) poolest. 

\vspace{2mm}
\hspace{5mm}
\textbf{Sarnased ON näiteks:} $3xy^{4}z^{9}$ , $-15xy^{4}z^{9}$ , $\dfrac{2}{7}xy^{4}z^{9}$\\
\hspace{5mm}
Pane tähele, et erinevad on vaid ees olevad numbrid!

\vspace{2mm}
\hspace{5mm}
\textbf{Sarnased EI OLE näiteks:} $3xyz$, $3abc$, $3xy^{2}z$, $12xyz^{2}$

\vspace{5mm}
\hspace{5mm}
Sarnaseid liikmeid saab koondada, mis teisisõnu tähendab, et kui liikmed erinevad vaid numbrite\\ \hspace{5mm} poolest, siis võib need numbrid omavahel kokku liita ja pärast need sarnased tähed järgi kirjutada.

\vspace{2mm}
\hspace{5mm}
\textbf{Näiteks:}

\vspace{2mm}
\hspace{5mm}
1) $3x+5x-9x+10x=(3+5-9+10)x=9x$

\vspace{2mm}
\hspace{5mm}
2) $12xy + 18z -2xy+9z-10x=(12-2)xy+(18+9)z-10x=10xy+27z-10x$
\end{flushleft}
}}}
\end{center}

\pagebreak

\begin{center}
\fbox{\fbox{\parbox{6.5in}{\centering
\begin{flushleft}

\vspace{2mm}
\hspace{5mm}
\textbf{\underline{Sulgude avamine}}

\vspace{2mm}
\hspace{5mm}
Mitmest erinevast üksliikmest koosnevat avaldist nimetatakse hulkliikmeks. Näiteks oli meil eelmisel\\ \hspace{5mm} lehel viimane vastus ($10xy+27z-10x$) hulkliige.


\vspace{2mm}
\hspace{5mm}
Võib esineda tehteid, kus sulgudes olev hulkliige on mõne arvuga läbi korrutatud. Sellisel juhul\\ \hspace{5mm} korrutatakse iga üksliige sulgude ees oleva arvuga läbi. Sellist protsessi nimetatakse sulgude\\ \hspace{5mm} avamiseks.

\vspace{2mm}
\hspace{5mm}
\textbf{Näiteks:}

\vspace{2mm}
\hspace{5mm}
$2(2a+b)-4(a-b)=2\cdot 2a + 2\cdot b +(-4) \cdot a + (-4)\cdot (-b) = 4a+2b-4a+4b=0a+6b=6b  $


\end{flushleft}
}}}
\end{center}

\vspace{0.5cm}

\textbf{Märkmed}\\
\vspace{2mm}
\begin{mdframed}[style=graphpaper]
\vspace{14cm}
\end{mdframed}
